\documentclass[11pt]{article}
\usepackage[margin=1in]{geometry}
\usepackage{amsmath,amssymb}
\usepackage{graphicx}
\usepackage{float}
\usepackage{booktabs}

\title{Reachability-Aware Guidance for Approach to a Tumbling Uncooperative Target with Time-Varying LOS Constraints}
\author{Author 1, Author 2, ... (placeholders)}
\date{}

\begin{document}
\maketitle

\begin{abstract}
This paper presents a reachability-aware guidance architecture for autonomous approach to a tumbling, uncooperative target under a rotating line-of-sight (LOS) docking corridor.
The LOS admissible set rotates with the target body frame, producing time-varying polyhedral constraints in the chaser's relative coordinates.
A safe-start region is constructed via two conservative criteria: (i)~directional per-constraint erosion, quantifying the margin consumed by rotation-induced drift before the thruster can arrest it, and (ii)~a synchronization range bound $r < 2a_{\max}/\omega_t^2$ ensuring the chaser can cancel the apparent rotational velocity without overshooting the hold point.
Closed-loop guidance uses a receding-horizon MPC controller with Clohessy-Wiltshire-Hill (CWH) prediction dynamics and explicit LOS corridor constraints embedded in the quadratic program.
Truth propagation uses the exact discrete CWH state-transition matrix with sub-stepping, ensuring physically honest feasibility claims.
Parametric sweeps over tumble rate (1--5~deg/s) and thrust authority (0.02--0.10~m/s$^2$) reveal the sensitivity of approach feasibility to the ratio $a_{\max}/\omega_t^2$, and an initial-range study confirms that close-range starts within the synchronization radius maintain zero LOS violations.
\end{abstract}

\noindent\textbf{Keywords:} proximity operations, uncooperative target, time-varying LOS corridor, safe-start region, synchronization bound, CWH guidance, MPC

\section{Introduction}
Autonomous rendezvous and docking with a tumbling uncooperative target requires guidance that respects time-varying geometric constraints imposed by the rotating docking corridor. The line-of-sight corridor rotates with the target body frame, producing time-varying feasibility boundaries in the chaser's coordinate system.

A critical pre-mission question is: \emph{from which initial states can the chaser safely approach and synchronize with the rotating hold point?} This paper addresses that question through a reachability-aware guidance architecture combining geometric safe-start region analysis, a CWH-based MPC controller with explicit LOS constraints, and physically honest truth propagation.

\section{Dynamics Model}
Both the guidance prediction model and truth propagation use three-dimensional CWH linearised relative dynamics in the LVLH frame:
\begin{align}
\ddot{x} &= 3n^2 x + 2n\dot{y} + a_x, \notag \\
\ddot{y} &= -2n\dot{x} + a_y, \\
\ddot{z} &= -n^2 z + a_z. \notag
\end{align}

Truth propagation uses the exact discrete-time state-transition matrix with 3 sub-steps per control interval. No reference blending, position projection, or velocity overrides are applied---the state evolves purely from CWH dynamics plus the commanded acceleration.

\section{Time-Varying LOS Corridor}
In body-frame coordinates $[x_B, y_B, z_B]^\top$, the LOS corridor is defined by five half-space inequalities $A_c \mathbf{p}_B \le b_c$, where $\mathbf{p}_B = R_z(-\omega_t t)\,\mathbf{p}$. With slopes $c_x = c_z = 1.5$ and half-widths $x_0 = z_0 = 2.5$~m, the corridor opens at approximately $56^\circ$ half-angle.

\section{Guidance Architecture}
The controller operates in three regimes:
\begin{enumerate}
\item \textbf{Far approach} ($r > r_{\mathrm{sync}}$): LVLH-frame PD tracking with velocity limiting.
\item \textbf{Close approach} ($r \le r_{\mathrm{sync}}$): body-frame PD tracking for co-rotation.
\item \textbf{Hold}: synchronized station-keeping with CWH gravity-gradient feedforward.
\end{enumerate}

A QP-based safety filter refines the nominal PD command with explicit LOS corridor constraints over a prediction horizon using CWH state-transition matrices.

\section{Safe-Start Region Analysis}
The safe-start analysis applies two conservative criteria:
\begin{itemize}
\item \textbf{Per-constraint erosion}: $\delta_i = (\dot{s}_i^-)^2 / (2 a_{\max})$
\item \textbf{Synchronization range bound}: $r < r_{\mathrm{sync}} = 2a_{\max}/\omega_t^2$
\end{itemize}

\section{Results}

\subsection{Sweep: Tumble Rate vs.\ Thrust Authority}
Table~\ref{tab:sweep} reports closed-loop outcomes for 15 combinations of $\omega_t$ and $a_{\max}$, starting from 195~m.

\begin{table}[H]
\centering
\caption{Parameter sweep: approach outcome vs.\ tumble rate and thrust authority (initial range 195~m).}
\label{tab:sweep}
\begin{tabular}{ccccccc}
\toprule
$\omega_t$ (deg/s) & $a_{\max}$ (m/s$^2$) & $r_{\mathrm{sync}}$ (m) & $r_{\min}$ (m) & $r_{\mathrm{final}}$ (m) & $\Delta v$ (m/s) & LOS viol. \\
\midrule
1 & 0.10 & 656.6 & 0.0 & 2.4 & 29.5 & 441 \\
1 & 0.05 & 328.3 & 143.7 & 274.4 & 15.0 & 803 \\
1 & 0.02 & 131.3 & 145.3 & 275.9 & 6.0 & 2313 \\
2 & 0.10 & 164.1 & 143.0 & 191.7 & 30.0 & 2615 \\
2 & 0.05 & 82.1 & 141.9 & 153.4 & 15.0 & 2021 \\
3 & 0.10 & 73.0 & 138.2 & 392.8 & 30.0 & 2070 \\
4 & 0.10 & 41.0 & 113.2 & 127.6 & 30.0 & 2003 \\
5 & 0.10 & 26.3 & 75.0 & 150.6 & 30.0 & 1996 \\
5 & 0.02 & 5.3 & 51.1 & 149.0 & 6.0 & 2060 \\
\bottomrule
\end{tabular}
\end{table}

\subsection{Initial-Range Study}
Starting from shorter ranges with the default scenario ($\omega_t = 1.72$~deg/s, $a_{\max} = 0.10$~m/s$^2$, $r_{\mathrm{sync}} = 222.2$~m):

\begin{table}[H]
\centering
\caption{Initial-range study results.}
\label{tab:range_study}
\begin{tabular}{cccccc}
\toprule
$r_0$ (m) & $r_{\min}$ (m) & $r_{\mathrm{final}}$ (m) & $\Delta v$ (m/s) & LOS viol. & Hold \\
\midrule
150 & 109.5 & 259.8 & 30.0 & 1807 & No \\
100 & 72.6 & 197.2 & 30.0 & 925 & No \\
50 & 29.9 & 29.9 & 29.8 & 0 & No \\
\bottomrule
\end{tabular}
\end{table}

The 50~m case achieves \textbf{zero LOS violations} with positive margin throughout, confirming the synchronization radius prediction.

\section{Conclusions}
A reachability-aware guidance architecture has been developed for approach to a tumbling target under a rotating LOS corridor. CWH-based MPC with explicit LOS constraints and three-regime tracking demonstrates feasible constraint-satisfying approach from within the synchronization radius. The transition to physically honest CWH truth dynamics revealed that earlier double-integrator implementations with reference blending produced artificially successful results. The 50~m initial-range case validates both the reachability analysis and the guidance architecture within the predicted feasible region.

\end{document}
