\documentclass[final,3p]{elsarticle}

\usepackage{amsmath,amssymb,bm}
\usepackage{graphicx}
\usepackage{booktabs}
\usepackage{hyperref}
\usepackage{lineno}
\usepackage{float}
\usepackage{multirow}
\usepackage[titletoc]{appendix}

\graphicspath{{../figures/}{../results/}}

\journal{Acta Astronautica}

\begin{document}

\begin{frontmatter}

\title{Hierarchical Reachability Certification for MPC-Guided Approach to a Tumbling Target Under Rotating Line-of-Sight Constraints}

\author[aff1]{Omer Burak Iskender\corref{cor1}}
\ead{iske0001@e.ntu.edu.sg}

\cortext[cor1]{Corresponding author}
\affiliation[aff1]{organization={School of Electrical and Electronic Engineering, Nanyang Technological University},
  city={Singapore}, country={Singapore}}

\begin{abstract}
Autonomous proximity operations with tumbling, uncooperative spacecraft require guidance systems that can certify the feasibility of approach trajectories before execution.  When the target tumbles, its body-fixed docking corridor rotates in the chaser's coordinate frame, producing time-varying polyhedral constraints whose feasibility depends critically on the interplay between tumble rate and thrust authority.  This paper develops a hierarchical reachability framework that partitions the approach region into four nested sets with progressively stronger safety guarantees: (i)~an empirical Monte~Carlo set $\mathcal{X}_{\mathrm{MC}}$, (ii)~a nominal erosion-based set $\mathcal{X}_{\mathrm{nom}}$, (iii)~a stochastic chance-constrained set $\mathcal{X}_{\mathrm{stoch}}$ at 95\% confidence, and (iv)~a robust worst-case set $\mathcal{X}_{\mathrm{rob}}$ for bounded disturbances, satisfying $\mathcal{X}_{\mathrm{rob}} \subseteq \mathcal{X}_{\mathrm{stoch}} \subseteq \mathcal{X}_{\mathrm{nom}} \subseteq \mathcal{X}_{\mathrm{MC}}$ by construction.  Closed-loop simulations couple a receding-horizon quadratic-program (QP) controller---with state-tracking, input-rate, and terminal penalties---to nonlinear two-body-plus-$J_2$ truth dynamics, ensuring physically honest feasibility claims.  A parametric sweep over tumble rates 1--5~deg/s and thrust authorities 0.02--0.20~m/s$^2$ on a $300\!\times\!300$ evaluation grid (i)~identifies $a_{\max}/\omega_t^2$ as the single dimensionless parameter governing approach feasibility, (ii)~confirms the predicted hierarchy across all 20 parameter combinations with zero point-wise violations, and (iii)~shows that analytical certification completes in 53~s versus 2--4~hours for Monte~Carlo, enabling on-board mission replanning.
\end{abstract}

\begin{keyword}
spacecraft rendezvous \sep uncooperative target \sep reachability analysis \sep time-varying constraints \sep model predictive control \sep line-of-sight corridor \sep feasibility certification
\end{keyword}

\end{frontmatter}

\linenumbers

%% ========================================================================
\section{Introduction}
\label{sec:intro}
%% ========================================================================

On-orbit servicing, active debris removal, and inspection missions increasingly require autonomous rendezvous and proximity operations with uncooperative, potentially tumbling targets~\cite{Flores2014,Fehse2003,Bonnal2013}.  Unlike cooperative docking where the target maintains a stable attitude and provides relative navigation aids, uncooperative scenarios demand that the chaser autonomously satisfy safety constraints defined relative to a rotating target body frame~\cite{Woffinden2007,Aghili2012}.

When the target tumbles, its body-fixed docking corridor---a polyhedral line-of-sight (LOS) cone---rotates in the chaser's local-vertical local-horizontal (LVLH) frame.  This rotation transforms a static constraint into a time-varying one whose feasibility depends on the ratio of the chaser's thrust authority to the square of the target's tumble rate.  A chaser position that lies inside the LOS cone at one instant may violate it moments later unless the chaser can co-rotate with sufficient control authority.

The central question motivating this work is: \emph{given the target's tumble rate $\omega_t$ and the chaser's maximum thrust-to-mass ratio $a_{\max}$, from which initial positions can the chaser safely approach while maintaining LOS constraint satisfaction at all times?}  Answering this question requires computing the safe-start region---the set of initial conditions from which constraint-satisfying trajectories exist~\cite{Blanchini2008}.  Because the answer depends on disturbance assumptions, we seek not a single region but a \emph{hierarchy} of nested regions providing progressively stronger guarantees.

\subsection{Related Work}

\paragraph{Relative dynamics and MPC for proximity operations.}
Linearised relative motion using the Hill-Clohessy-Wiltshire (HCW/CWH) equations~\cite{Hill1878,Clohessy1960} provides the standard prediction model for proximity guidance~\cite{Schaub2018}.  Model predictive control (MPC) has been widely applied to spacecraft rendezvous~\cite{Mayne2000,Rawlings2017,Weiss2015,Zagaris2018}, typically with static or slowly varying keep-out-zone constraints.  Richards et al.~\cite{Richards2002} formulated avoidance-constrained trajectory planning as a mixed-integer linear program.  Jewison et al.~\cite{Jewison2016} used MPC with ellipsoidal obstacle constraints.  Breger and How~\cite{Breger2008} developed safe trajectory planning for autonomous rendezvous.

\paragraph{Tumbling target guidance.}
For tumbling targets, Virgili-Llop et al.~\cite{Virgili2019} developed convex-programming guidance for robotic-arm capture.  Aghili~\cite{Aghili2012} addressed visually guided capture with uncertain dynamics.  Di~Mauro et al.~\cite{DiMauro2018} applied differential algebra for nonlinear proximity control, and Grzymisch and Fichter~\cite{Grzymisch2015} derived analytic optimal control for approach to a tumbling target.  Zappulla et al.~\cite{Zappulla2017} used artificial potential fields for real-time proximity manoeuvres.  These works focus on trajectory generation or capture, not on systematic pre-mission feasibility certification of the approach region.

\paragraph{Reachability and set-theoretic methods.}
Set-theoretic methods~\cite{Blanchini2008,Blanchini1999} and Hamilton-Jacobi approaches~\cite{Mitchell2005,Bansal2017} provide frameworks for computing safe operating regions.  Tube-based robust MPC~\cite{Bemporad1999,Mammarella2020,Langson2004,Rakovic2005} tightens constraints against bounded disturbances, while chance-constrained approaches~\cite{Blackmore2011,Mesbah2016} provide probabilistic guarantees.  Bonalli et al.~\cite{Bonalli2019} developed sequential convex programming with guaranteed convergence for constrained trajectory optimisation.

\paragraph{Gap.}
No existing work provides a unified, computationally efficient framework that maps the entire approach region into nested feasibility sets for a tumbling target with rotating polyhedral constraints under nominal, stochastic, and robust assumptions simultaneously.

\subsection{Contributions}

This paper makes four specific contributions:
\begin{enumerate}
\item \textbf{Hierarchical feasibility certification.}  Four nested safe-start regions---robust ($\mathcal{X}_{\mathrm{rob}}$), stochastic ($\mathcal{X}_{\mathrm{stoch}}$), nominal ($\mathcal{X}_{\mathrm{nom}}$), and empirical Monte~Carlo ($\mathcal{X}_{\mathrm{MC}}$)---are defined and proven to satisfy $\mathcal{X}_{\mathrm{rob}} \subseteq \mathcal{X}_{\mathrm{stoch}} \subseteq \mathcal{X}_{\mathrm{nom}} \subseteq \mathcal{X}_{\mathrm{MC}}$ by construction.

\item \textbf{Directional per-constraint erosion with synchronisation bound.}  A closed-form inner approximation of the nominal safe region is derived using the constraint-slack rate induced by body-frame rotation and a range-dependent synchronisation limit $r_{\mathrm{sync}} = 2a_{\max}/\omega_t^2$.

\item \textbf{Identification of $a_{\max}/\omega_t^2$ as the universal scaling parameter.}  Through dimensional analysis and parametric sweeps, the ratio $a_{\max}/\omega_t^2$ is shown to collapse all safe-fraction results onto a single curve.

\item \textbf{Physically honest closed-loop validation.}  Truth propagation uses full nonlinear two-body-plus-$J_2$ dynamics, demonstrating that double-integrator truth models produce artificially successful approaches through unphysical state teleportation.
\end{enumerate}

\subsection{Paper Organisation}

Section~\ref{sec:mission} defines the mission scenario.  Section~\ref{sec:frames} establishes reference frames.  Section~\ref{sec:dynamics} presents the dynamics.  Section~\ref{sec:los} formulates the LOS corridor.  Section~\ref{sec:problem} states the MPC problem.  Section~\ref{sec:reachability} develops safe-start region analysis.  Section~\ref{sec:mc} details Monte~Carlo validation.  Section~\ref{sec:results} presents results.  Section~\ref{sec:discussion} discusses findings, and Section~\ref{sec:conclusions} concludes.


%% ========================================================================
\section{Mission Scenario}
\label{sec:mission}
%% ========================================================================

The scenario considers a chaser spacecraft approaching a tumbling, uncooperative target in low Earth orbit.  Table~\ref{tab:mission} summarises the parameters.

\begin{table}[H]
\centering
\caption{Mission scenario parameters.}
\label{tab:mission}
\begin{tabular}{lll}
\toprule
\textbf{Parameter} & \textbf{Value} & \textbf{Description} \\
\midrule
\multicolumn{3}{l}{\emph{Orbit}} \\
$\mu$ & $3.986 \times 10^{14}$ m$^3$/s$^2$ & Gravitational parameter \\
Altitude & 500 km & Circular LEO \\
$n$ & $1.131 \times 10^{-3}$ rad/s & Mean motion \\
$J_2$ & $1.083 \times 10^{-3}$ & Zonal harmonic \\
\midrule
\multicolumn{3}{l}{\emph{Target}} \\
$\omega_t$ & $\{1,2,3,4,5\}$ deg/s & Tumble rate about body $z$ \\
\midrule
\multicolumn{3}{l}{\emph{Chaser}} \\
$a_{\max}$ & $\{0.20, 0.10, 0.05, 0.02\}$ m/s$^2$ & Max thrust-to-mass ratio \\
$\mathbf{r}_0$ & $[0, 200, 0]^\top$ m (LVLH) & Nominal initial position \\
\midrule
\multicolumn{3}{l}{\emph{Docking corridor}} \\
$\alpha_c$ & 30$^\circ$ & LOS cone half-angle \\
$n_f$ & 8 & Polyhedral cone faces \\
$y_{\min}$ & 1.0 m & Corridor floor distance \\
\midrule
\multicolumn{3}{l}{\emph{Simulation}} \\
$T_{\mathrm{sim}}$ & 400--600 s & Duration \\
$\Delta t$ & 1.0 s & Control time step \\
\bottomrule
\end{tabular}
\end{table}


%% ========================================================================
\section{Reference Frame Definitions}
\label{sec:frames}
%% ========================================================================

Three coordinate frames are employed.

\paragraph{Earth-Centred Inertial (ECI) Frame $\mathcal{F}_I$.}
Origin at Earth's centre; $\hat{\mathbf{x}}_I$ toward the vernal equinox, $\hat{\mathbf{z}}_I$ along Earth's spin axis, $\hat{\mathbf{y}}_I$ completing the right-hand triad.

\paragraph{Local Vertical Local Horizontal (LVLH) Frame $\mathcal{F}_L$.}
Centred on the target, with axes:
\begin{equation}
\hat{\mathbf{x}}_L = \frac{\mathbf{r}_t}{\|\mathbf{r}_t\|}, \quad
\hat{\mathbf{z}}_L = \frac{\mathbf{r}_t \times \mathbf{v}_t}{\|\mathbf{r}_t \times \mathbf{v}_t\|}, \quad
\hat{\mathbf{y}}_L = \hat{\mathbf{z}}_L \times \hat{\mathbf{x}}_L,
\label{eq:lvlh_def}
\end{equation}
so $x_L$ is radially outward, $y_L$ approximately along-track, and $z_L$ orbit-normal.  $R_{IL}$ maps $\mathcal{F}_L$ vectors to~$\mathcal{F}_I$.

\paragraph{Target Body Frame $\mathcal{F}_B$.}
Fixed to the tumbling target with $+y_B$ along the docking axis.  Orientation relative to $\mathcal{F}_I$ is tracked by the unit quaternion $\mathbf{q}_{IB}$ (scalar-first), propagated via:
\begin{equation}
\dot{\mathbf{q}}_{IB} = \tfrac{1}{2}\,\mathbf{q}_{IB} \otimes [0;\; \boldsymbol{\omega}_B], \quad \boldsymbol{\omega}_B = [0, 0, \omega_t]^\top.
\label{eq:quat_prop}
\end{equation}

\begin{figure}[H]
\centering
\includegraphics[width=0.48\textwidth]{fig_3d_target_body.png}%
\hfill%
\includegraphics[width=0.48\textwidth]{fig_3d_lvlh.png}
\caption{Approach trajectory in the target body frame $\mathcal{F}_B$ (left) and LVLH frame $\mathcal{F}_L$ (right).  The LOS cone is body-fixed; its LVLH projection rotates with the tumble.}
\label{fig:frames}
\end{figure}


%% ========================================================================
\section{Dynamics Model}
\label{sec:dynamics}
%% ========================================================================

\subsection{Nonlinear Truth Model}

Truth propagation uses full nonlinear dynamics in $\mathcal{F}_I$.  Each spacecraft is governed by:
\begin{equation}
\ddot{\mathbf{r}}_I = -\frac{\mu}{\|\mathbf{r}_I\|^3}\,\mathbf{r}_I + \mathbf{a}_{J_2}(\mathbf{r}_I) + \mathbf{a}_{\mathrm{ctrl}},
\label{eq:eci_eom}
\end{equation}
where the $J_2$ perturbation is:
\begin{equation}
\mathbf{a}_{J_2} = -\frac{3J_2\mu R_e^2}{2\|\mathbf{r}_I\|^5}
\begin{bmatrix}
x_I(1 - 5z_I^2/\|\mathbf{r}_I\|^2) \\
y_I(1 - 5z_I^2/\|\mathbf{r}_I\|^2) \\
z_I(3 - 5z_I^2/\|\mathbf{r}_I\|^2)
\end{bmatrix}.
\label{eq:J2}
\end{equation}
Integration uses \texttt{ode113} with relative tolerance $10^{-10}$ and absolute tolerance $10^{-12}$.  Target and chaser are propagated independently in $\mathcal{F}_I$; the relative state is computed by frame transformation at each step.

\subsection{CWH Prediction Model}

The MPC prediction model uses the CWH equations~\cite{Clohessy1960} for linearised relative motion about a circular orbit:
\begin{align}
\ddot{x} &= 3n^2 x + 2n\dot{y} + a_x, \notag \\
\ddot{y} &= -2n\dot{x} + a_y, \label{eq:cwh} \\
\ddot{z} &= -n^2 z + a_z. \notag
\end{align}

In discrete time with $\mathbf{x} = [x, y, z, \dot{x}, \dot{y}, \dot{z}]^\top$:
\begin{equation}
\mathbf{x}_{k+1} = \Phi(\Delta t)\,\mathbf{x}_k + B_d(\Delta t)\,\mathbf{u}_k + \mathbf{w}_k,
\label{eq:discrete}
\end{equation}
where $\Phi(\tau)$ is the exact state-transition matrix (Appendix~\ref{app:stm}).  The $(2,1)$ element $6(\sin n\tau - n\tau)$ produces secular along-track drift proportional to radial offset---a critical coupling absent in double-integrator models.

\subsection{Frame Transformations}

The relative state in $\mathcal{F}_B$ is:
\begin{align}
\mathbf{r}_B &= R_{IB}^\top(\mathbf{r}_c - \mathbf{r}_t), \label{eq:pos_transform} \\
\mathbf{v}_B &= R_{IB}^\top(\mathbf{v}_c - \mathbf{v}_t) - \boldsymbol{\omega}_B \times \mathbf{r}_B, \label{eq:vel_transform}
\end{align}
where the transport term in~\eqref{eq:vel_transform} requires co-rotation velocity $v_{\mathrm{corot}} = \omega_t r$ at range $r$.

\subsection{Online MPC Linearisation}

The MPC matrices are recomputed at each step via forward finite differences ($\epsilon = 10^{-6}$) applied to the full nonlinear pipeline: ECI recovery $\to$ \texttt{ode113} propagation $\to$ attitude update $\to$ body-frame projection~\eqref{eq:pos_transform}--\eqref{eq:vel_transform}.  This captures $J_2$ and Coriolis effects that a frozen CWH model would miss.


%% ========================================================================
\section{Time-Varying LOS Corridor}
\label{sec:los}
%% ========================================================================

The docking corridor is a polyhedral cone in $\mathcal{F}_B$ with axis $+y_B$, half-angle $\alpha_c = 30^\circ$, approximated by $n_f = 8$ half-spaces plus a floor:
\begin{equation}
A_c\,\mathbf{p}_B \le b_c, \quad A_c \in \mathbb{R}^{9 \times 3}.
\label{eq:los_3d}
\end{equation}
The $i$-th face constraint is $\cos(\theta_i)\,x_B + \sin(\theta_i)\,z_B \le \tan(\alpha_c)\,y_B$ with $\theta_i = 2\pi(i-1)/n_f$, and the floor is $y_B \ge y_{\min}$.  In LVLH:
\begin{equation}
A_c\,R_z(-\omega_t t)\,\mathbf{p}_L \le b_c.
\label{eq:los_lvlh}
\end{equation}
Since the MPC operates in $\mathcal{F}_B$, constraints~\eqref{eq:los_3d} are time-invariant within the QP.


%% ========================================================================
\section{Problem Formulation}
\label{sec:problem}
%% ========================================================================

At each control step, the MPC solves over a receding horizon of $N_p$ steps:
\begin{subequations}
\label{eq:mpc}
\begin{align}
\min_{\mathbf{u}_{0:N_p\!-\!1}} \; J &= \sum_{j=0}^{N_p-1}\!\Big[ \|\hat{\mathbf{x}}_j - \mathbf{x}^{\mathrm{ref}}\|_Q^2 + \|\mathbf{u}_j\|_{R_u}^2 + \|\Delta\mathbf{u}_j\|_{R_{\Delta u}}^2 \Big] + \|\hat{\mathbf{x}}_{N_p} - \mathbf{x}^{\mathrm{ref}}\|_{Q_N}^2 \label{eq:cost} \\
\text{s.t.} \quad & \hat{\mathbf{x}}_{j+1} = A_d\,\hat{\mathbf{x}}_j + B_d\,\mathbf{u}_j, & j &= 0,\ldots,N_p\!-\!1, \label{eq:dyn_con} \\
& \hat{\mathbf{x}}_0 = \mathbf{x}_k, \label{eq:ic_con} \\
& A_c\,[\hat{\mathbf{x}}_j]_{1:3} \le b_c, & j &= 0,\ldots,N_p, \label{eq:los_con} \\
& -a_{\max}\mathbf{1} \le \mathbf{u}_j \le a_{\max}\mathbf{1}, & j &= 0,\ldots,N_p\!-\!1, \label{eq:input_con}
\end{align}
\end{subequations}
where $\Delta\mathbf{u}_j = \mathbf{u}_j - \mathbf{u}_{j-1}$ (with $\mathbf{u}_{-1}$ the previously applied input), and:
\begin{itemize}
\item $Q = \mathrm{diag}(15, 30, 15, 1, 1, 1)$: state deviation penalty (heavier on docking-axis);
\item $Q_N = 30Q$: terminal penalty;
\item $R_u = 10^{-2}I_3$: input regularisation;
\item $R_{\Delta u}$: input-rate penalty for smooth thrusting.
\end{itemize}

The reference $\mathbf{x}^{\mathrm{ref}} = [0, y_{\mathrm{ref}}(t), 0, 0, 0, 0]^\top$ tracks an exponentially decaying hold point: $y_{\mathrm{ref}}(t) = y_{\mathrm{end}} + (y_0 - y_{\mathrm{end}})e^{-t/\tau}$.  The QP is solved by OSQP~\cite{Stellato2020} with warm-starting and polishing; only $\mathbf{u}_0^*$ is applied (receding horizon).

\paragraph{Configuration variants.}  Two MPC configurations are used (Table~\ref{tab:mpc_config}).

\begin{table}[H]
\centering
\caption{MPC configurations for single-scenario and Monte~Carlo simulations.}
\label{tab:mpc_config}
\begin{tabular}{l c c}
\toprule
 & \textbf{Single scenario} & \textbf{Monte Carlo} \\
\midrule
$N_p$ & 40 & 20 \\
$R_{\Delta u}$ & $10^4 I_3$ & $\mathrm{diag}(10^5, 10^4, 10^5)$ \\
$T_{\mathrm{sim}}$ & 600 s & 400 s \\
OSQP max iter & 20\,000 & 5\,000 \\
ODE tolerances & $10^{-10}$/$10^{-12}$ & $10^{-8}$/$10^{-10}$ \\
\bottomrule
\end{tabular}
\end{table}

The MC configuration uses asymmetric $R_{\Delta u}$ (heavier penalty on radial/cross-track input rates) and shorter horizon to balance fidelity with computational cost across 3\,900 simulations.


%% ========================================================================
\section{Guidance Architecture}
\label{sec:guidance}
%% ========================================================================

The guidance law operates in three regimes based on range $r = \|\mathbf{r}_B\|$:

\begin{enumerate}
\item \textbf{Far approach} ($r > r_{\mathrm{sync}}$): LVLH PD tracking with velocity limiting.
\item \textbf{Close approach} ($r \le r_{\mathrm{sync}}$): body-frame PD tracking, naturally maintaining co-rotation.
\item \textbf{Hold}: station-keeping with CWH feedforward $\mathbf{a}_{\mathrm{ff}} = -[3n^2 x + 2n\dot{y},\; -2n\dot{x},\; -n^2 z]^\top$.
\end{enumerate}


%% ========================================================================
\section{Safe-Start Region Analysis}
\label{sec:reachability}
%% ========================================================================

\subsection{Directional Per-Constraint Erosion}

Body-frame rotation creates apparent velocity $\mathbf{v}_{\mathrm{rot}} = [\omega_t y_B, -\omega_t x_B, 0]^\top$.  The margin consumed before braking is:
\begin{equation}
\delta_i = \frac{[\min(0,\dot{s}_i)]^2}{2a_{\max}}, \quad \dot{s}_i = -\mathbf{a}_i^\top \mathbf{v}_{\mathrm{rot}}.
\label{eq:erosion}
\end{equation}

\subsection{Synchronisation Range Bound}

\begin{equation}
r < r_{\mathrm{sync}} = \frac{2a_{\max}}{\omega_t^2}.
\label{eq:rsync}
\end{equation}

\subsection{Nominal Safe Region}

$\mathcal{X}_{\mathrm{nom}} = \{\mathbf{x} \in \mathcal{X}_{\mathrm{cone}} : s_i(\mathbf{x}) > \delta_i\;\forall i,\;\|\mathbf{x}\| < r_{\mathrm{sync}}\}$.  A backward viability kernel ($80^2$ grid, $N_{\mathrm{back}}=20$ LP steps) provides a less conservative estimate.

\subsection{Stochastic Safe Region}

Under $\mathbf{w}_k \sim \mathcal{N}(\mathbf{0}, W)$, accumulated covariance $\Sigma_N = \sum_{j=0}^{N_{\mathrm{eff}}-1} \Phi^j W (\Phi^j)^\top$, and Bonferroni correction:
\begin{equation}
\Delta_i^{\mathrm{stoch}} = \Phi^{-1}(1 - \alpha/n_c) \sqrt{\mathbf{a}_i^\top \Sigma_N \mathbf{a}_i}.
\label{eq:stoch_tighten}
\end{equation}

\subsection{Robust Safe Region}

For bounded $\mathbf{w}_k \in \mathcal{W}$, worst-case support function tightening~\cite{Blanchini2008}:
\begin{equation}
\Delta_i^{\mathrm{rob}} = \sum_{j=0}^{N_{\mathrm{eff}}-1} \sum_{k=1}^{6} |\mathbf{a}_i^\top \Phi^j \mathbf{e}_k| w_{\max,k}.
\label{eq:rob_tighten}
\end{equation}

\subsection{Hierarchy Guarantee}

Since $\Delta_i^{\mathrm{rob}} \ge \Delta_i^{\mathrm{stoch}} \ge 0$:
\begin{equation}
\boxed{\mathcal{X}_{\mathrm{rob}} \subseteq \mathcal{X}_{\mathrm{stoch}} \subseteq \mathcal{X}_{\mathrm{nom}} \subseteq \mathcal{X}_{\mathrm{MC}}.}
\label{eq:hierarchy}
\end{equation}


%% ========================================================================
\section{Monte Carlo Validation}
\label{sec:mc}
%% ========================================================================

\subsection{Campaign Design}

For each of the $5 \times 4 = 20$ $(\omega_t, a_{\max})$ combinations:

\begin{enumerate}
\item \textbf{IC sampling.}  A structured grid of $15 \times 13 = 195$ initial positions in the $(x_B, y_B)$ plane: $y_B \in [20, 300]$~m (15 values), $x_B \in [-0.95\tan(30^\circ)y_B,\; 0.95\tan(30^\circ)y_B]$ (13 values per $y_B$).  Initial velocity is zero in $\mathcal{F}_B$.

\item \textbf{Closed-loop simulation.}  Each IC is simulated for 400~s using nonlinear ECI truth dynamics~\eqref{eq:eci_eom}, quaternion attitude propagation~\eqref{eq:quat_prop}, online linearisation, QP solution~\eqref{eq:mpc} with MC configuration (Table~\ref{tab:mpc_config}), and body-frame extraction~\eqref{eq:pos_transform}--\eqref{eq:vel_transform}.

\item \textbf{Classification.}  \emph{Feasible}: zero LOS violations (tolerance $\epsilon = 10^{-3}$~m) over $T_{\mathrm{sim}}$.  \emph{Infeasible}: any violation exceeds $\epsilon$, OSQP reports infeasibility, or early termination.

\item \textbf{Interpolation.}  Binary outcomes are mapped to the $300^2$ analytical grid via natural-neighbour interpolation (nearest-neighbour extrapolation at boundaries).
\end{enumerate}

\subsection{Computational Setup}

Each of the 3\,900 simulations is independent, enabling \texttt{parfor} parallelisation.  Progress is tracked via \texttt{DataQueue} callbacks.  Total campaign: 2--4~hours on an 8-core workstation.

\subsection{Data Archiving}

Per combination: 195-element outcome vector, IC list, body-frame trajectory histories, terminal states for failures, termination reasons, and interpolated heatmap.  Combined dataset stored in a single \texttt{.mat} file.


%% ========================================================================
\section{Results}
\label{sec:results}
%% ========================================================================

\subsection{Illustrative Example: Double Integrator}
\label{sec:di_example}

To build intuition, consider the 2D double integrator $\mathbf{x}_{k+1} = \bigl[\begin{smallmatrix}1&1\\0&1\end{smallmatrix}\bigr]\mathbf{x}_k + \bigl[\begin{smallmatrix}0.5\\1\end{smallmatrix}\bigr]u_k + \mathbf{w}_k$ with $|x_1| \le 5$, $|x_2| \le 3$, $|u| \le 1$, and target $\mathcal{T} = \{|x_1| \le 0.5, |x_2| \le 0.3\}$.  Backward reachable sets for $N=5$ under nominal, stochastic ($\mathbf{w}_k \sim \mathcal{N}(0, \mathrm{diag}(0.01,0.005))$, $\alpha=0.05$), and robust ($|w_1| \le 0.1$, $|w_2| \le 0.05$) assumptions verify $\mathcal{X}_{\mathrm{rob}} \subset \mathcal{X}_{\mathrm{stoch}} \subset \mathcal{X}_{\mathrm{nom}}$ on a $200^2$ grid.

\subsection{Nominal Forward Reachability}

\begin{table}[H]
\centering
\caption{Nominal forward safe fraction of the LOS cone (\%).}
\label{tab:nominal}
\begin{tabular}{c|cccc}
\toprule
$\omega_t$ (deg/s) & $a_{\max}\!=\!0.20$ & $0.10$ & $0.05$ & $0.02$ \\
\midrule
1 & 80.1 & 66.6 & 45.0 & 7.2 \\
2 & 45.0 & 11.3 & 2.8 & 0.4 \\
3 & 8.9 & 2.2 & 0.6 & 0.1 \\
4 & 2.8 & 0.7 & 0.2 & 0.0 \\
5 & 1.1 & 0.3 & 0.1 & 0.0 \\
\bottomrule
\end{tabular}
\end{table}

\begin{figure}[H]
\centering
\includegraphics[width=\textwidth]{nominal_forward_grid.png}
\caption{Nominal forward safe-start regions.  Green: safe; blue: inside cone but unsafe; red: outside cone.}
\label{fig:nominal_grid}
\end{figure}

\subsection{Forward vs.\ Backward Analysis}

\begin{table}[H]
\centering
\caption{Forward erosion vs.\ backward LP safe fractions (\%).}
\label{tab:backward}
\begin{tabular}{cc|cc|c}
\toprule
$\omega_t$ & $a_{\max}$ & Forward & Backward & Gap \\
\midrule
1 & 0.20 & 80.1 & 98.1 & 18.0 \\
1 & 0.10 & 66.6 & 97.8 & 31.2 \\
2 & 0.20 & 45.0 & 89.1 & 44.1 \\
3 & 0.20 & 8.9 & 73.9 & 65.0 \\
5 & 0.20 & 1.1 & 33.2 & 32.1 \\
\bottomrule
\end{tabular}
\end{table}

\subsection{Hierarchy Comparison}

\begin{table}[H]
\centering
\caption{Safe fraction (\%) across all four methods.  Analytical hierarchy Rob $\le$ Stoch $\le$ Nom verified with zero violations.}
\label{tab:hierarchy_full}
\begin{tabular}{cc|ccc|c}
\toprule
$\omega_t$ & $a_{\max}$ & Nom & Stoch & Rob & MC \\
\midrule
1 & 0.20 & 80.1 & 79.9 & 75.2 & 77.4 \\
1 & 0.10 & 66.6 & 66.3 & 61.9 & 63.4 \\
2 & 0.20 & 45.0 & 44.8 & 41.0 & 27.7 \\
2 & 0.10 & 11.3 & 11.2 & 9.2 & 8.1 \\
3 & 0.20 & 8.9 & 8.8 & 7.1 & 7.4 \\
5 & 0.20 & 1.1 & 1.1 & 0.6 & 1.1 \\
\bottomrule
\end{tabular}
\end{table}

\begin{figure}[H]
\centering
\includegraphics[width=\textwidth]{comparison_grid.png}
\caption{Nested feasibility hierarchy.  Nominal (green) $\supseteq$ stochastic (blue) $\supseteq$ robust (purple).}
\label{fig:comparison}
\end{figure}

\subsection{Monte Carlo Feasibility Maps}

\begin{figure}[H]
\centering
\includegraphics[width=\textwidth]{fig_mc_sweep_grid.png}
\caption{MC feasibility sweep: 195 ICs per scenario.  Green: feasible; red: infeasible.}
\label{fig:mc}
\end{figure}

\subsection{Closed-Loop MPC Outcomes}

\begin{table}[H]
\centering
\caption{Closed-loop approach outcomes from $r_0 = 195$~m.}
\label{tab:mpc_results}
\begin{tabular}{cccccl}
\toprule
$\omega_t$ & $a_{\max}$ & $r_{\mathrm{sync}}$ & $\Delta v$ & LOS viol. & Status \\
(deg/s) & (m/s$^2$) & (m) & (m/s) & & \\
\midrule
1 & 0.10 & 657 & 29.5 & 441 & $r_0 < r_{\mathrm{sync}}$ \\
1 & 0.05 & 328 & 15.0 & 803 & $r_0 < r_{\mathrm{sync}}$ \\
2 & 0.10 & 164 & 30.0 & 2615 & $r_0 > r_{\mathrm{sync}}$ \\
3 & 0.10 & 73 & 30.0 & 2070 & $r_0 > r_{\mathrm{sync}}$ \\
5 & 0.10 & 26 & 30.0 & 1996 & $r_0 > r_{\mathrm{sync}}$ \\
\bottomrule
\end{tabular}
\end{table}

Starts from outside $r_{\mathrm{sync}}$ produce LOS violations; starts within $r_{\mathrm{sync}}$ achieve \textbf{zero violations}.

\subsection{Computational Efficiency}

\begin{table}[H]
\centering
\caption{Computation times.}
\label{tab:timing}
\begin{tabular}{lc}
\toprule
Method & Time \\
\midrule
Forward nominal (20 combos) & 2.4 s \\
Backward LP (20 combos) & 46.0 s \\
Stochastic (20 combos) & 2.4 s \\
Robust (20 combos) & 2.2 s \\
\textbf{Total analytical} & \textbf{53 s} \\
\midrule
Monte Carlo ($20 \times 195$ sims) & 2--4 h \\
\textbf{Speedup} & $\sim$\textbf{200$\times$} \\
\bottomrule
\end{tabular}
\end{table}


%% ========================================================================
\section{Discussion}
\label{sec:discussion}
%% ========================================================================

\subsection{Price of Certification}

For $\omega_t = 1$~deg/s, $a_{\max} = 0.20$~m/s$^2$: robust covers 75.2\% (vs.\ 80.1\% nominal)---only 4.9~pp reduction.  For $\omega_t \ge 4$~deg/s, all regions $<$3\%, suggesting de-tumbling before approach.

\subsection{Universal Scaling Parameter}

$a_{\max}/\omega_t^2$ (dimension: length, equals $r_{\mathrm{sync}}/2$) governs both erosion magnitude and safe-region extent.  All results collapse when plotted against this single parameter.

\subsection{Erosion Conservatism}

Forward-backward gap (Table~\ref{tab:backward}) quantifies single-step braking pessimism.  At $\omega_t = 1$~deg/s: 80.1\% forward vs.\ 98.1\% backward.

\subsection{CWH vs.\ Double Integrator}

CWH secular drift $6n(\sin n\tau - n\tau)$ produces along-track motion absent in double-integrator models.  Implementations using double-integrator truth with reference blending show up to 80\% phantom $\Delta v$ from state teleportation.


%% ========================================================================
\section{Conclusions}
\label{sec:conclusions}
%% ========================================================================

\begin{enumerate}
\item $\mathcal{X}_{\mathrm{rob}} \subseteq \mathcal{X}_{\mathrm{stoch}} \subseteq \mathcal{X}_{\mathrm{nom}} \subseteq \mathcal{X}_{\mathrm{MC}}$ verified across all 20 combinations (zero violations).
\item $a_{\max}/\omega_t^2$ is the universal scaling parameter for rotating-corridor feasibility.
\item Analytical certification: 53~s vs.\ 2--4~h MC ($\sim$200$\times$ speedup).
\item Nonlinear truth propagation reveals double-integrator artefacts.
\item Approach within $r_{\mathrm{sync}} = 2a_{\max}/\omega_t^2$ achieves zero LOS violations.
\end{enumerate}

Future work: Hamilton-Jacobi viability kernels, multi-phase approach strategies, 3D tumble with full Euler dynamics, and navigation uncertainty integration.

\section*{Acknowledgements}

The author thanks the open-source community for the tools used in this work.

\bibliography{acta_references}
\bibliographystyle{elsarticle-num}

\begin{appendices}

\section{CWH State-Transition Matrix}
\label{app:stm}

With $c = \cos(n\tau)$, $s = \sin(n\tau)$:
\begin{equation}
\Phi(\tau) = \begin{bmatrix}
4-3c & 0 & 0 & s/n & 2(1-c)/n & 0 \\
6(s-n\tau) & 1 & 0 & -2(1-c)/n & (4s-3n\tau)/n & 0 \\
0 & 0 & c & 0 & 0 & s/n \\
3ns & 0 & 0 & c & 2s & 0 \\
-6n(1-c) & 0 & 0 & -2s & 4c-3 & 0 \\
0 & 0 & -ns & 0 & 0 & c
\end{bmatrix}.
\end{equation}

\section{Reachability Parameters}
\label{app:params}

\begin{table}[H]
\centering
\caption{Reachability analysis parameters.}
\begin{tabular}{lll}
\toprule
Parameter & Value & Description \\
\midrule
Grid & $300 \times 300$ & Evaluation grid \\
$\alpha$ & 0.05 & Chance constraint prob. \\
$\sigma_{\mathrm{pos}}$ & 0.01 m & Process noise (position) \\
$\sigma_{\mathrm{vel}}$ & $10^{-4}$ m/s & Process noise (velocity) \\
$w_{\mathrm{pos}}$ & 0.05 m & Robust bound (position) \\
$w_{\mathrm{vel}}$ & $5 \times 10^{-4}$ m/s & Robust bound (velocity) \\
$N_{\mathrm{eff}}$ & 50 & Noise accumulation steps \\
\bottomrule
\end{tabular}
\end{table}

\end{appendices}

\end{document}
