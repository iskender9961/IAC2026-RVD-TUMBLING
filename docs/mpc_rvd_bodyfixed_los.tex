% ------------------------------------------------------------------
%  MPC-Based Rendezvous with Tumbling Target: Body-Fixed LOS Constraints
%  Author: Iskender O.B.
%  Compile: pdflatex mpc_rvd_bodyfixed_los.tex
% ------------------------------------------------------------------
\documentclass[11pt,a4paper]{article}

% ---- packages ----------------------------------------------------
\usepackage[margin=2.5cm]{geometry}
\usepackage{amsmath,amssymb}
\usepackage{graphicx}
\usepackage{hyperref}
\usepackage{booktabs}
\usepackage{bm}
\usepackage{float}

% ---- graphics guard: allow missing figures without error ----------
\let\OrigIncludeGraphics\includegraphics
\renewcommand{\includegraphics}[2][]{%
  \IfFileExists{#2}{%
    \OrigIncludeGraphics[#1]{#2}%
  }{%
    \begin{center}%
      \fbox{\parbox{0.7\textwidth}{\centering\vspace{2cm}%
        \textbf{[Figure placeholder]}\\[4pt]%
        \texttt{#2} not found.%
        \vspace{2cm}}}%
    \end{center}%
  }%
}

% ---- macros -------------------------------------------------------
\newcommand{\bx}{\bm{x}}
\newcommand{\bu}{\bm{u}}
\newcommand{\bz}{\bm{z}}
\newcommand{\br}{\bm{r}}
\newcommand{\bv}{\bm{v}}
\newcommand{\ba}{\bm{a}}
\newcommand{\bomega}{\bm{\omega}}
\newcommand{\bq}{\bm{q}}
\newcommand{\R}{\mathbb{R}}
\newcommand{\norm}[1]{\left\lVert #1 \right\rVert}
\newcommand{\cross}{\times}
\DeclareMathOperator{\diag}{diag}

% ---- document -----------------------------------------------------
\begin{document}

% ====================================================================
\title{MPC-Based Rendezvous with Tumbling Target:\\
       Body-Fixed LOS Constraints}
\author{Iskender O.B.}
\date{}
\maketitle

% ====================================================================
\begin{abstract}
This document describes a model-predictive-control (MPC) formulation for
autonomous rendezvous and docking with a tumbling target spacecraft.
The key feature is that the line-of-sight (LOS) keep-out cone constraint
is expressed in the \emph{target body frame}, so that it naturally
rotates with the tumbling target.  The plant propagation uses nonlinear
ECI dynamics with $J_2$ perturbations, while the MPC prediction model
uses a linear time-varying (LTV) discrete system re-linearised at every
control step.  The resulting quadratic programme is solved with OSQP.
\end{abstract}

\tableofcontents
\newpage

% ====================================================================
\section{Scenario Description}\label{sec:scenario}
% ====================================================================

\subsection{Orbit and Spacecraft}

The \textbf{target} spacecraft occupies a 500\,km circular low-Earth
orbit (LEO) with semi-major axis
\begin{equation}\label{eq:sma}
  a = R_E + 500\;\text{km} = 6\,878\,137\;\text{m},
\end{equation}
where $R_E = 6\,378\,137$\,m is the Earth equatorial radius.  The orbit
is equatorial (inclination $i = 0^{\circ}$).

The \textbf{chaser} spacecraft starts in the vicinity of the target and
performs the final approach manoeuvre under MPC guidance.

\subsection{Coordinate Frames}\label{sec:frames}

Three right-handed frames are used throughout:

\begin{enumerate}
  \item \textbf{ECI} (Earth-Centred Inertial) ---
        Standard $J2000$ frame.  All truth dynamics are propagated in
        ECI.

  \item \textbf{LVLH} (Local Vertical Local Horizontal) ---
        Centred on the target centre of mass.  Axes:
        \begin{align}
          \hat{\bx}_L &= \frac{\br_{\text{tgt}}}
                            {\norm{\br_{\text{tgt}}}}\;
                         (\text{radial, outward}), \notag\\
          \hat{\bz}_L &= \frac{\br_{\text{tgt}} \cross
                                \bv_{\text{tgt}}}
                            {\norm{\br_{\text{tgt}} \cross
                                    \bv_{\text{tgt}}}}\;
                         (\text{orbit normal}), \label{eq:lvlh}\\
          \hat{\bm{y}}_L &= \hat{\bz}_L \cross \hat{\bx}_L\;
                         (\text{along-track direction}). \notag
        \end{align}
        The rotation matrix $R_{\text{ECI}\leftarrow\text{LVLH}}$ has
        columns $[\hat{\bx}_L\;\hat{\bm{y}}_L\;\hat{\bz}_L]$.

  \item \textbf{Target Body (TB)} ---
        Fixed to the target rigid body.  At $t = 0$ the TB frame is
        aligned with LVLH\@.  For $t > 0$ the target tumbles with a
        constant body-fixed angular velocity (see
        Section~\ref{sec:tumble}).
\end{enumerate}

\subsection{Tumble Model}\label{sec:tumble}

The target attitude is parameterised by the unit quaternion
$\bq = [q_0,\; q_1,\; q_2,\; q_3]^T$ (scalar-first convention).  The
body angular velocity is assumed constant:
\begin{equation}\label{eq:omega_body}
  \bomega_{\text{body}} = \begin{bmatrix} 0 \\ 0 \\ \omega_z \end{bmatrix}
  \quad\text{rad/s},
\end{equation}
which represents a pure spin about the body $z$-axis.

The quaternion kinematics are
\begin{equation}\label{eq:qdot}
  \dot{\bq} = \frac{1}{2}\,\Omega(\bomega_{\text{body}})\,\bq,
  \qquad
  \Omega(\bomega) =
  \begin{bmatrix}
     0        & -\omega_x & -\omega_y & -\omega_z \\
     \omega_x &  0        &  \omega_z & -\omega_y \\
     \omega_y & -\omega_z &  0        &  \omega_x \\
     \omega_z &  \omega_y & -\omega_x &  0
  \end{bmatrix}.
\end{equation}
With constant $\bomega_{\text{body}}$ the solution is a simple matrix
exponential; in the simulation, however, equation~\eqref{eq:qdot} is
integrated alongside the translational dynamics to maintain a single
integrator state.

\subsection{Docking Axis}

The docking port is located on the $+\hat{\bm{y}}_T$ axis of the target
body frame.  All line-of-sight constraints are therefore defined with
respect to this axis (Section~\ref{sec:los}).


% ====================================================================
\section{Truth Dynamics (Plant Model)}\label{sec:truth}
% ====================================================================

Both the target and chaser are propagated in ECI using the same
nonlinear equations of motion:
\begin{equation}\label{eq:eom}
  \dot{\br} = \bv, \qquad
  \dot{\bv} = \ba_{2\text{b}}(\br) + \ba_{J_2}(\br) + \ba_{\text{ctrl}},
\end{equation}
where the control acceleration $\ba_{\text{ctrl}}$ is zero for the
target and nonzero for the chaser.

\subsection{Two-Body Acceleration}

\begin{equation}\label{eq:2body}
  \ba_{2\text{b}}(\br) = -\frac{\mu}{\norm{\br}^3}\,\br,
\end{equation}
with the gravitational parameter
$\mu = 3.986 \times 10^{14}$\,m$^3$\,s$^{-2}$.

\subsection{$J_2$ Perturbation}

Denoting $r = \norm{\br}$ and the ECI components
$\br = [x,\;y,\;z]^T$, the standard $J_2$ acceleration is
\begin{equation}\label{eq:j2}
  \ba_{J_2}(\br) =
  -\frac{3\,\mu\,J_2\,R_E^2}{2\,r^5}
  \begin{bmatrix}
    x\!\left(1 - 5\,\dfrac{z^2}{r^2}\right) \\[6pt]
    y\!\left(1 - 5\,\dfrac{z^2}{r^2}\right) \\[6pt]
    z\!\left(3 - 5\,\dfrac{z^2}{r^2}\right)
  \end{bmatrix},
\end{equation}
with $J_2 = 1.08263 \times 10^{-3}$.

\subsection{Numerical Constants}

For reference the three key constants are collected here:
\begin{equation}\label{eq:constants}
  \mu = 3.986 \times 10^{14}\;\text{m}^3/\text{s}^2, \qquad
  R_E = 6\,378\,137\;\text{m}, \qquad
  J_2 = 1.08263 \times 10^{-3}.
\end{equation}

\subsection{Integration}

The plant model is integrated with MATLAB's \texttt{ode113}
(variable-step Adams--Bashforth--Moulton) using tight tolerances
($\texttt{RelTol} = 10^{-12}$, $\texttt{AbsTol} = 10^{-14}$).
Integration proceeds one MPC control interval $\Delta t$ at a time so
that the piece-wise-constant control input is applied exactly.


% ====================================================================
\section{MPC State in Target Body Frame}\label{sec:tb_state}
% ====================================================================

The MPC operates on the six-dimensional state
\begin{equation}\label{eq:mpc_state}
  \bx = \begin{bmatrix} \br_{\text{TB}} \\ \bv_{\text{TB}} \end{bmatrix}
      \in \R^6,
\end{equation}
where $\br_{\text{TB}}$ and $\bv_{\text{TB}}$ are the relative position
and velocity of the chaser with respect to the target, expressed in the
target body frame.

\subsection{Rotation Matrices}

\paragraph{LVLH from orbit state.}
From the target ECI position $\br_{\text{tgt}}$ and velocity
$\bv_{\text{tgt}}$, the DCM
\begin{equation}\label{eq:R_eci_lvlh}
  R_{\text{ECI}\leftarrow\text{LVLH}}
  = \bigl[\;\hat{\bx}_L \;\;\hat{\bm{y}}_L \;\;\hat{\bz}_L\;\bigr]
\end{equation}
is constructed as in equation~\eqref{eq:lvlh}.

\paragraph{Target body from quaternion.}
The rotation matrix corresponding to the attitude quaternion $\bq$ is
\begin{equation}\label{eq:R_eci_tb}
  R_{\text{ECI}\leftarrow\text{TB}} = R(\bq),
\end{equation}
computed via the standard quaternion-to-DCM formula (scalar-first).

\subsection{Position Transformation}

\begin{equation}\label{eq:r_tb}
  \br_{\text{TB}} = R_{\text{TB}\leftarrow\text{ECI}}\,
                     \bigl(\br_{\text{cha}} - \br_{\text{tgt}}\bigr)
                   = R(\bq)^T
                     \bigl(\br_{\text{cha}} - \br_{\text{tgt}}\bigr).
\end{equation}

\subsection{Velocity Transformation -- Transport Theorem}

Because the TB frame rotates with angular velocity
$\bomega_{\text{body}}$ relative to inertial space, the time derivative
of a vector expressed in TB picks up a transport term.  The relative
velocity in the body frame is
\begin{equation}\label{eq:v_tb}
  \boxed{
  \bv_{\text{TB}} = R_{\text{TB}\leftarrow\text{ECI}}\,
                     \bigl(\bv_{\text{cha}} - \bv_{\text{tgt}}\bigr)
                   \;-\; \bomega_{\text{body}} \cross \br_{\text{TB}}.
  }
\end{equation}

\paragraph{Explanation of the transport term.}
Let $\br_{\text{rel}}^{\text{ECI}} = \br_{\text{cha}} - \br_{\text{tgt}}$
denote the relative position in ECI\@.  Its time derivative
\emph{as seen in the rotating TB frame} is
\[
  \left.\frac{d\br_{\text{rel}}}{dt}\right|_{\text{TB}}
  = \left.\frac{d\br_{\text{rel}}}{dt}\right|_{\text{ECI}}
    - \bomega_{\text{body}} \cross \br_{\text{rel}}^{\text{TB}}.
\]
Here the first term on the right is simply
$R_{\text{TB}\leftarrow\text{ECI}}(\bv_{\text{cha}} - \bv_{\text{tgt}})$
(the inertial relative velocity projected into TB), while the second
term accounts for the apparent velocity induced by the frame rotation.
Omitting this cross-product would introduce a velocity bias proportional
to $\omega_z\,\norm{\br_{\text{TB}}}$, which grows as the chaser
approaches from large stand-off distances.


% ====================================================================
\section{MPC Optimisation Problem}\label{sec:mpc}
% ====================================================================

\subsection{Decision Variables}

Over a prediction horizon of $N$ steps the decision variables are
\begin{itemize}
  \item States:
    $\bx_0, \bx_1, \ldots, \bx_N \in \R^6$,
  \item Inputs:
    $\bu_0, \bu_1, \ldots, \bu_{N-1} \in \R^3$
    (thrust acceleration in the TB frame).
\end{itemize}

\subsection{Cost Function}

\begin{equation}\label{eq:cost}
  J = \sum_{k=0}^{N-1}
      \Bigl[
        (\bx_k - \bx_{\text{ref},k})^T Q\,(\bx_k - \bx_{\text{ref},k})
        + \bu_k^T R_u\,\bu_k
        + \Delta\bu_k^T R_{\Delta u}\,\Delta\bu_k
      \Bigr]
      + (\bx_N - \bx_{\text{ref},N})^T Q_N\,(\bx_N - \bx_{\text{ref},N}),
\end{equation}
where the input increment is
\begin{equation}\label{eq:du}
  \Delta\bu_k = \bu_k - \bu_{k-1}, \qquad
  \bu_{-1} \triangleq \bu_{\text{prev}}
\end{equation}
(the last applied input from the previous MPC solve).

The weighting matrices are:
\begin{itemize}
  \item $Q \in \R^{6\times6}$ --- stage state cost (positive
        semi-definite),
  \item $Q_N \in \R^{6\times6}$ --- terminal state cost (positive
        semi-definite),
  \item $R_u \in \R^{3\times3}$ --- input magnitude cost (positive
        definite),
  \item $R_{\Delta u} \in \R^{3\times3}$ --- input-rate cost (positive
        definite), penalising thruster cycling.
\end{itemize}

\subsection{Dynamics Constraint (LTV)}\label{sec:ltv}

At each MPC call the nonlinear dynamics
$\bx^{+} = f(\bx, \bu)$
are linearised about the current state $\bar{\bx}$ and zero input via
\emph{finite differences} to obtain
\begin{equation}\label{eq:dynamics}
  \bx_{k+1} = A_d\,\bx_k + B_d\,\bu_k,
  \qquad k = 0,\ldots,N-1,
\end{equation}
where $A_d \in \R^{6\times6}$ and $B_d \in \R^{6\times3}$ are the
discrete-time Jacobians valid for one control interval.  The
linearisation is updated (\emph{re-linearised}) at every outer MPC
step, making the prediction model \emph{linear time-varying}.

\subsection{Input Bounds}

Element-wise box constraints on the thrust acceleration:
\begin{equation}\label{eq:ubox}
  -u_{\max}\,\bm{1}_3 \;\le\; \bu_k \;\le\; u_{\max}\,\bm{1}_3,
  \qquad k = 0,\ldots,N-1.
\end{equation}

\subsection{Body-Fixed LOS Cone Constraint}\label{sec:los}

The docking corridor is a cone about the $+\hat{\bm{y}}_T$ axis with
half-angle $\alpha$.  In the TB frame the chaser position is
$\br_{\text{TB}} = [x_T,\; y_T,\; z_T]^T$.  The exact second-order
cone constraint is
\begin{equation}\label{eq:soc}
  \sqrt{x_T^2 + z_T^2} \;\le\; \tan(\alpha)\;y_T,
  \qquad y_T \ge y_{\min},
\end{equation}
which ensures the chaser lies inside the cone and ahead of the minimum
distance plane.

\paragraph{Polyhedral approximation.}
To keep the problem as a QP (rather than SOCP), the cone is
approximated by $m$ half-planes.  Define
\begin{equation}\label{eq:theta_i}
  \theta_i = \frac{2\pi\,(i-1)}{m}, \qquad i = 1,\ldots,m,
  \qquad k = \tan(\alpha).
\end{equation}
Then the polyhedral inner approximation of~\eqref{eq:soc} is
\begin{equation}\label{eq:poly_cone}
  \boxed{
  \cos\theta_i\;x_T + \sin\theta_i\;z_T \;\le\; k\;y_T,
  \qquad i = 1,\ldots,m,}
\end{equation}
together with
\begin{equation}\label{eq:ymin}
  y_T \;\ge\; y_{\min}.
\end{equation}
Each inequality is linear in $\bx_k$, so it enters the QP as a standard
row of the constraint matrix.

\paragraph{Key advantage.}
Because the state $\bx$ is already expressed in the target body frame,
the cone constraint~\eqref{eq:poly_cone} is \emph{time-invariant} in
the MPC formulation.  The cone ``automatically'' rotates with the
tumbling target; no explicit rotation of the constraint normals is
needed at each prediction step.

\subsection{Reference Trajectory}\label{sec:ref}

The reference state $\bx_{\text{ref},k}$ places the chaser on the
$+\hat{\bm{y}}_T$ axis at a decaying stand-off distance:
\begin{equation}\label{eq:ref}
  \br_{\text{ref}}(t_k) = \begin{bmatrix} 0 \\ d(t_k) \\ 0 \end{bmatrix},
  \qquad
  d(t_k) = d_{\infty} + (d_0 - d_{\infty})\,e^{-\lambda\,t_k},
  \qquad
  \bv_{\text{ref}}(t_k) = \begin{bmatrix} 0 \\ \dot{d}(t_k) \\ 0 \end{bmatrix},
\end{equation}
where $d_0$ is the initial stand-off, $d_{\infty}$ the terminal
(docking) distance, and $\lambda > 0$ the decay rate.


% ====================================================================
\section{OSQP QP Implementation}\label{sec:osqp}
% ====================================================================

The MPC~\eqref{eq:cost}--\eqref{eq:ymin} is cast into the canonical
OSQP form
\begin{equation}\label{eq:qp}
  \min_{\bz}\;\frac{1}{2}\,\bz^T P\,\bz + \bm{q}^T\bz
  \qquad\text{subject to}\qquad
  \bm{l} \;\le\; A_{\text{qp}}\,\bz \;\le\; \bm{u},
\end{equation}
with the stacked decision vector
\begin{equation}\label{eq:zvec}
  \bz = \begin{bmatrix}
    \bx_0 \\ \bx_1 \\ \vdots \\ \bx_N \\
    \bu_0 \\ \bu_1 \\ \vdots \\ \bu_{N-1}
  \end{bmatrix}
  \;\in\;\R^{6(N+1)+3N}.
\end{equation}

\subsection{Cost Matrix $P$}

The Hessian is block-diagonal (up to the $\Delta u$ cross-terms):
\begin{equation}\label{eq:P}
  P = \diag\!\bigl(\underbrace{Q,\ldots,Q}_{N},\;Q_N,\;
              \underbrace{\tilde{R},\ldots,\tilde{R}}_{N}\bigr)
      + P_{\Delta u},
\end{equation}
where $\tilde{R} = R_u + 2\,R_{\Delta u}$ on the diagonal blocks and
$P_{\Delta u}$ adds the off-diagonal $-R_{\Delta u}$ coupling between
consecutive $\bu_k$ blocks.  The linear term $\bm{q}$ absorbs the
reference offsets ($-2\,Q\,\bx_{\text{ref},k}$, etc.) and the
$\bu_{\text{prev}}$ contribution from $\Delta\bu_0$.

\subsection{Constraint Matrix $A_{\text{qp}}$}

The constraint matrix is assembled row-by-row from three groups:

\begin{enumerate}
  \item \textbf{Dynamics equalities} ($6N$ rows). \\
    For $k = 0,\ldots,N-1$:
    \[
      A_d\,\bx_k + B_d\,\bu_k - \bx_{k+1} = \bm{0}.
    \]
    Encoded as $l_i = u_i = 0$ (equality).

  \item \textbf{Input box constraints} ($6N$ rows: 3 lower + 3 upper per step). \\
    $-u_{\max} \le [\bu_k]_j \le u_{\max}$, $j = 1,2,3$, for each $k$.

  \item \textbf{LOS polyhedral constraints} ($m \cdot (N+1) + (N+1)$ rows). \\
    For every prediction step $k$ and every facet $i$:
    \[
      \cos\theta_i\;[\bx_k]_1 + \sin\theta_i\;[\bx_k]_3
        - k\;[\bx_k]_2 \;\le\; 0,
    \]
    plus $[\bx_k]_2 \ge y_{\min}$.
\end{enumerate}

\subsection{Warm-Starting and Sparsity}

The OSQP solver object is created once and stored persistently across
MPC steps.  At each step only the entries that change are updated:
\begin{itemize}
  \item $\bm{q}$ (linear cost, depends on the current reference and
        $\bu_{\text{prev}}$),
  \item Non-zero entries of $A_{\text{qp}}$ (the $A_d$, $B_d$ blocks
        change when re-linearised),
  \item Bound vectors $\bm{l}$, $\bm{u}$ (if $y_{\min}$ or $u_{\max}$
        are scheduled).
\end{itemize}
All matrices ($P$, $A_{\text{qp}}$) are stored in compressed sparse
column (CSC) format, which OSQP requires.  Warm-starting the primal and
dual variables from the previous solution typically halves the iteration
count.


% ====================================================================
\section{Verification}\label{sec:verification}
% ====================================================================

A suite of unit-level and integration-level tests is executed before
every simulation campaign.

\subsection{Test 1: $J_2$ Magnitude Sanity}

At the reference orbit altitude the ratio of $J_2$ to two-body
acceleration should be of order $10^{-3}$:
\begin{equation}
  \frac{\norm{\ba_{J_2}}}{\norm{\ba_{2\text{b}}}}
  \approx \frac{3}{2}\,J_2\,\frac{R_E^2}{r^2}
  \approx 1.1 \times 10^{-3}.
\end{equation}
The test asserts that the computed ratio lies in
$[5\times10^{-4},\;5\times10^{-3}]$.

\subsection{Test 2: TB Equals LVLH at $t = 0$}

At the initial epoch the attitude quaternion is set to the identity
($\bq_0 = [1,\;0,\;0,\;0]^T$), so the body frame coincides with LVLH\@.
The test checks
\begin{equation}
  \norm{R_{\text{ECI}\leftarrow\text{TB}}(t_0)
       - R_{\text{ECI}\leftarrow\text{LVLH}}(t_0)}_F
  < \epsilon,
\end{equation}
with $\epsilon = 10^{-12}$.

\subsection{Test 3: On-Axis Feasibility}

A point on the docking axis $\br_{\text{TB}} = [0,\;y,\;0]^T$ with
$y > y_{\min}$ must satisfy \emph{all} $m$ facets of the polyhedral
cone~\eqref{eq:poly_cone}:
\[
  \cos\theta_i \cdot 0 + \sin\theta_i \cdot 0
  = 0 \;\le\; k\,y.
\]
Since $k = \tan\alpha > 0$ and $y > 0$ the inequality is trivially
satisfied.  The test confirms OSQP returns a feasible solution for a
single-step problem with $\bx_0$ on the axis.

\subsection{Test 4: Tiny Cone Infeasibility}

When the half-angle is set to an extremely small value
($\alpha = 0.01^{\circ}$) and the chaser starts off-axis, the
polyhedral constraints become so tight that no feasible trajectory
exists within the thrust budget.  The test asserts that OSQP returns
status \texttt{primal\_infeasible} (or dual infeasible, depending on
the formulation).

\subsection{Test 5: Cone Rotates with Target Body}

Two snapshots of the constraint normals are compared:
\begin{enumerate}
  \item At $t = 0$ (TB aligned with LVLH), the cone axis in LVLH is
        $+\hat{\bm{y}}_L$.
  \item After a quarter-revolution of the tumble ($t = \pi/(2\omega_z)$),
        the cone axis in LVLH has rotated by $90^{\circ}$ about
        $\hat{\bz}_L$.
\end{enumerate}
Since the MPC state is in TB, the constraint rows do \emph{not} change;
only the state mapping does.  The test verifies that a point that was
inside the cone at $t = 0$ (expressed in LVLH) is \emph{outside} at
$t = \pi/(2\omega_z)$ when the same LVLH coordinates are transformed to
TB\@.


% ====================================================================
\section{Simulation Results}\label{sec:results}
% ====================================================================

\subsection{3D Approach Trajectory}

Figure~\ref{fig:3d} shows the chaser trajectory in the target body
frame together with the LOS cone and the three coordinate frames (ECI,
LVLH, TB).

\begin{figure}[H]
  \centering
  \includegraphics[width=0.8\textwidth]{fig_3d_approach.png}
  \caption{Three-dimensional approach trajectory in the target body frame.
           The translucent cone represents the polyhedral LOS constraint;
           the chaser path (blue) remains inside the cone throughout the
           manoeuvre.  Coordinate triads for ECI (red), LVLH (green), and
           TB (cyan) are shown at selected epochs.}
  \label{fig:3d}
\end{figure}

\subsection{Time Histories}

Figure~\ref{fig:time} presents the time evolution of the along-docking-axis
distance $y_T$, the radial deviation $\sqrt{x_T^2 + z_T^2}$, and the
control acceleration magnitude $\norm{\bu}$.

\begin{figure}[H]
  \centering
  \includegraphics[width=0.8\textwidth]{fig_time_histories.png}
  \caption{Time histories.  \textbf{Top:} distance along the docking
           axis ($y_T$), showing the exponential decay towards the
           terminal hold point.  \textbf{Middle:} radial deviation from
           the docking axis, bounded by the cone constraint.
           \textbf{Bottom:} thrust acceleration magnitude, remaining
           within the saturation limit $u_{\max}$ at all times.}
  \label{fig:time}
\end{figure}


% ====================================================================
\section*{Acknowledgements}
% ====================================================================
This work uses the OSQP solver (Stellato et al., 2020, \textit{Mathematical Programming Computation}, 12(4), 637--672) for the embedded quadratic programme.

\end{document}
