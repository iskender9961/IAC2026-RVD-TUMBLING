% MPC Formulation for Proximity Approach to a Tumbling Target
% Auto-generated from run_sweep_mpc_90cases.py

\documentclass[11pt]{article}
\usepackage{amsmath,amssymb,bm}
\usepackage{booktabs}
\usepackage{geometry}
\geometry{margin=2.5cm}

\newcommand{\vx}{\bm{x}}
\newcommand{\vu}{\bm{u}}
\newcommand{\vr}{\bm{r}}
\newcommand{\vv}{\bm{v}}
\newcommand{\vs}{\bm{s}}

\begin{document}

\section*{MPC Formulation: Proximity Approach to a Tumbling Target}

%% ============================================================
\subsection*{1. Relative Dynamics (CWH / Hill--Clohessy--Wiltshire)}

The chaser--target relative motion is modelled with the linearised
Clohessy--Wiltshire--Hill (CWH) equations in the target-centred LVLH frame.
The continuous-time state $\vx = [x,\,y,\,z,\,\dot x,\,\dot y,\,\dot z]^\top$
evolves as
\begin{equation}\label{eq:cwh}
\dot{\vx} = A_c\,\vx + B_c\,\vu,
\end{equation}
where
\[
A_c =
\begin{bmatrix}
0 & 0 & 0 & 1 & 0 & 0 \\
0 & 0 & 0 & 0 & 1 & 0 \\
0 & 0 & 0 & 0 & 0 & 1 \\
3n^2 & 0 & 0 & 0 & 2n & 0 \\
0 & 0 & 0 & -2n & 0 & 0 \\
0 & 0 & -n^2 & 0 & 0 & 0
\end{bmatrix},
\qquad
B_c =
\begin{bmatrix}
0_{3\times3} \\[2pt] I_{3}
\end{bmatrix},
\]
$n = 1.1\times10^{-3}$\,rad/s is the orbital mean motion, and
$\vu = [a_x,\,a_y,\,a_z]^\top$ is the control acceleration
(m/s$^2$).

The system is discretised via exact matrix-exponential zero-order hold:
\begin{equation}
\vx_{k+1} = A_d\,\vx_k + B_d\,\vu_k,
\qquad
\begin{bmatrix} A_d & B_d \\ 0 & I \end{bmatrix}
= \exp\!\left(
\begin{bmatrix} A_c & B_c \\ 0 & 0 \end{bmatrix} \Delta t
\right),
\end{equation}
with $\Delta t_{\mathrm{MPC}} = 1.0$\,s for the prediction model
and $\Delta t_{\mathrm{truth}} = 0.1$\,s for the closed-loop truth propagation
(10 sub-steps per MPC interval).


%% ============================================================
\subsection*{2. Frame Transformation: LVLH $\leftrightarrow$ Body}

The target tumbles about the $z$-axis at constant angular rate
$\omega_t$.  At time $t$ the rotation angle is $\theta(t) = \omega_t\,t$.
The body-to-LVLH rotation is $R_z(\theta)$ and we define the
$6{\times}6$ state-rotation matrix
\begin{equation}
T(\theta) =
\begin{bmatrix}
R_z(-\theta) & 0 \\ 0 & R_z(-\theta)
\end{bmatrix},
\qquad
R_z(\theta) =
\begin{bmatrix}
\cos\theta & -\sin\theta & 0 \\
\sin\theta &  \cos\theta & 0 \\
0 & 0 & 1
\end{bmatrix},
\end{equation}
so that $\vx_B = T(\theta)\,\vx_L$ maps the LVLH state to the body frame.

\textbf{Note:} $T(\theta)$ uses $R_z(-\theta) = R_z(\theta)^\top$ because
it transforms \emph{from} LVLH \emph{to} body.  The cost is evaluated in
body frame (penalising body-frame position/velocity errors), but the
dynamics and the decision variables live in LVLH.


%% ============================================================
\subsection*{3. MPC Cost Function}

At each MPC step the controller solves a finite-horizon QP over
$N=30$ steps.  The decision variables are
\[
\bigl\{\vx_1,\ldots,\vx_N,\;
       \vu_0,\ldots,\vu_{N-1},\;
       \vs_0,\ldots,\vs_{N-1}\bigr\},
\]
where $\vs_j \in \mathbb{R}^5_{\ge 0}$ are slack variables for the
soft LOS constraints.

The cost is
\begin{equation}\label{eq:cost}
\boxed{
J = \underbrace{\sum_{j=1}^{N-1}
       \vx_j^\top T_j^\top\, Q\, T_j\, \vx_j
     + \vx_N^\top T_N^\top\, P\, T_N\, \vx_N}_{\text{state cost (body frame)}}
   + \underbrace{\sum_{j=0}^{N-1}
       r_{\Delta u}\,\|\Delta\vu_j\|^2}_{\text{control smoothing}}
   + \underbrace{\sum_{j=0}^{N-1}
       w_s\bigl(\|\vs_j\|^2 + \bm{1}^\top \vs_j\bigr)}_{\text{constraint softening}}
}
\end{equation}
where:
\begin{itemize}
\item $T_j = T\bigl(\omega_t(t_{\mathrm{now}} + j\,\Delta t)\bigr)$
      rotates LVLH state to body frame at the predicted time of step~$j$.
\item $\Delta\vu_j = \vu_j - \vu_{j-1}$, with $\vu_{-1} = \vu_{\mathrm{prev}}$
      (previous applied control).
\item $P$ is the DARE terminal cost satisfying
      $P = A_d^\top P\, A_d - A_d^\top P\, B_d
       (R + B_d^\top P\, B_d)^{-1} B_d^\top P\, A_d + Q$,
      with $R = r_{\Delta u}\,I_3$.
\item $w_s = w_{\mathrm{slack}} \cdot (a_{\max}/0.10)$ scales the
      penalty to the thrust level for numerical conditioning.
\end{itemize}

\paragraph{Cost variant A (position $+$ $\Delta u$):}
\[
Q_A = \operatorname{diag}(5,\;50,\;5,\;0,\;0,\;0),
\qquad
r_{\Delta u} = 30.
\]
Only body-frame position error is penalised; the $y_B$ axis (docking
direction) is weighted $10\times$ more than $x_B$ and $z_B$.
There is \emph{no} velocity penalty and \emph{no} direct control penalty.

\paragraph{Cost variant B (position $+$ velocity $+$ $\Delta u$):}
\[
Q_B = \operatorname{diag}(5,\;50,\;5,\;2,\;10,\;2),
\qquad
r_{\Delta u} = 30.
\]
Same as Cost~A but adds body-frame velocity error penalties.

\bigskip
\begin{center}
\begin{tabular}{lcc}
\toprule
Weight & Cost A & Cost B \\
\midrule
$q_{x_B}$ & 5 & 5 \\
$q_{y_B}$ & 50 & 50 \\
$q_{z_B}$ & 5 & 5 \\
$q_{\dot x_B}$ & 0 & 2 \\
$q_{\dot y_B}$ & 0 & 10 \\
$q_{\dot z_B}$ & 0 & 2 \\
$r_{\Delta u}$ & 30 & 30 \\
$w_{\mathrm{slack}}$ & $10^4$ & $10^4$ \\
Direct $\|\vu\|^2$ penalty & 0 & 0 \\
\bottomrule
\end{tabular}
\end{center}


%% ============================================================
\subsection*{4. Line-of-Sight (LOS) Cone Constraint}

The approach corridor is defined in the \emph{target body frame} as a
polyhedral cone (frustum) around the $+y_B$ docking axis.  Five
half-plane constraints enforce:
\begin{align}
y_B &\ge y_{\min}, \label{eq:los1}\\
|x_B| &\le x_0 + c_x\,(y_B - y_{\min}), \label{eq:los2}\\
|z_B| &\le z_0 + c_z\,(y_B - y_{\min}), \label{eq:los3}
\end{align}
with parameters $y_{\min} = 0.5$\,m (hold radius),
$x_0 = z_0 = 2.5$\,m (cone base half-width),
$c_x = c_z = 1.5$ (cone slope).

In matrix form (position-only, body frame):
\begin{equation}
H_{\mathrm{LOS}}\,\vr_B \le \bm{b}_{\mathrm{LOS}},
\end{equation}
\[
H_{\mathrm{LOS}} =
\begin{bmatrix}
 0 & -1 &  0 \\
 1 & -c_x &  0 \\
-1 & -c_x &  0 \\
 0 & -c_z &  1 \\
 0 & -c_z & -1
\end{bmatrix},
\quad
\bm{b}_{\mathrm{LOS}} =
\begin{bmatrix}
-y_{\min} \\
x_0 - c_x\,y_{\min} \\
x_0 - c_x\,y_{\min} \\
z_0 - c_z\,y_{\min} \\
z_0 - c_z\,y_{\min}
\end{bmatrix}
=
\begin{bmatrix}
-0.50 \\
1.75 \\
1.75 \\
1.75 \\
1.75
\end{bmatrix}.
\]

Since the LVLH position is $\vr_L$ and $\vr_B = R_z(-\theta)^\top\,\vr_L
= R_z(\theta)^\top\,\vr_L$, the constraint at prediction step~$j$ becomes
\begin{equation}\label{eq:los_lvlh}
\underbrace{H_{\mathrm{LOS}}\,R_z(\theta_j)^\top}_{=:\,H_j}\;\vr_{L,j}
\;\le\;
\bm{b}_{\mathrm{LOS}} - \epsilon_{\mathrm{margin}} + \vs_j,
\qquad \vs_j \ge 0,
\end{equation}
where $\theta_j = \omega_t\,(t_{\mathrm{now}} + j\,\Delta t)$ and
$\epsilon_{\mathrm{margin}} = 0.05$\,m provides inter-sample rotation safety.
The slack $\vs_j$ is penalised in the cost~\eqref{eq:cost} via quadratic and
linear terms.


%% ============================================================
\subsection*{5. Input (Thrust) Constraint}

Component-wise box constraint:
\begin{equation}
-a_{\max} \le u_i \le a_{\max},
\qquad i \in \{x,\,y,\,z\},
\end{equation}
where $a_{\max} \in \{0.10,\;0.06,\;0.02\}$\,m/s$^2$.

There is \emph{no} total impulse constraint; the only limit on fuel
expenditure comes from the finite simulation time (600\,s) and the
per-component thrust bound.

%% ============================================================
\subsection*{6. Synchronisation Radius}

A key feasibility indicator is the \emph{synchronisation radius},
\begin{equation}
r_{\mathrm{sync}} = \frac{2\,a_{\max}}{\omega_t^2}.
\end{equation}
This is the maximum distance at which the chaser can sustain
co-rotation with the tumbling target.  When the initial range
$y_{\mathrm{lim}} > r_{\mathrm{sync}}$, docking is generally infeasible.


%% ============================================================
\subsection*{7. QP Formulation Summary}

Collecting all terms, the QP solved at each MPC step is:
\begin{align}
\min_{\vx,\,\vu,\,\vs} \quad & J \text{ as in~\eqref{eq:cost}} \\[4pt]
\text{s.t.} \quad
& \vx_{j} = A_d\,\vx_{j-1} + B_d\,\vu_j,
  && j = 1,\ldots,N, \quad \vx_0 = \vx_{\mathrm{current}} \\
& -a_{\max}\,\bm{1} \le \vu_j \le a_{\max}\,\bm{1},
  && j = 0,\ldots,N{-}1 \\
& H_j\,\vr_{L,j} \le \bm{b}_{\mathrm{LOS}} - \epsilon + \vs_j,
  && j = 1,\ldots,N \\
& \vs_j \ge 0,
  && j = 1,\ldots,N
\end{align}

\paragraph{Solver:} OSQP (Operator Splitting QP),
$\epsilon_{\mathrm{abs}} = \epsilon_{\mathrm{rel}} = 10^{-4}$,
max iter $= 4000$, polish~$=$~true.
The Hessian is normalised by $\max|\mathrm{diag}(P)|$ for conditioning.

\paragraph{Fallback:} If the QP is infeasible, a braking controller is applied:
$\vu = -\frac{a_{\max}}{2}\,\hat{\vr} - \frac{a_{\max}}{2}\,\hat{\vv}$
(decelerate towards origin and damp velocity).

\paragraph{Docking criterion:}
$\|\vr\| \le 0.5$\,m and $\|\vv\| \le 0.05$\,m/s.

\end{document}
