\documentclass[11pt]{article}
\usepackage[margin=1in]{geometry}
\usepackage{amsmath}
\usepackage{amssymb}
\usepackage{enumitem}
\usepackage{graphicx}
\usepackage{booktabs}
\usepackage{float}
\usepackage{hyperref}
\usepackage{xcolor}

\title{Reachability-Aware Start-Region Feasibility Maps for Proximity Guidance\\
under Rotating Line-of-Sight Constraints}
\author{IAC 2026 --- Working Draft}
\date{}

\begin{document}
\maketitle

%% ====================================================================
\section{Introduction and Motivation}
%% ====================================================================

Rendezvous, proximity operations, and docking (RPOD) with a non-cooperative
tumbling target remain among the most challenging autonomous guidance problems
in spaceflight.  When the target rotates, the line-of-sight (LOS) feasibility
corridor---which constrains the chaser to a polyhedral cone aligned with the
docking port---sweeps in inertial space.  Standard practice evaluates safety
via Monte Carlo campaigns, which provide statistical coverage but no
structural guarantees about which initial conditions are persistently
feasible under the rotating constraint set.

This paper proposes a \emph{reachability-inspired} method for computing
certified (or conservatively inner-approximated) safe start regions that
remain within the rotating LOS corridor throughout the approach, as a function
of two key mission parameters:
\begin{itemize}[leftmargin=1.4em]
  \item target tumble rate $\omega_t$,
  \item chaser thrust-to-mass ratio $a_{\max}$.
\end{itemize}
For each ($\omega_t$, $a_{\max}$) pair, the method yields a \emph{nested}
family of safe regions (higher $a_{\max}$ $\Rightarrow$ larger safe set)
visualised as overlay feasibility maps in the target body frame.

The main contribution is therefore a \textbf{parametric, reachability-based
feasibility map that goes beyond Monte Carlo} and provides actionable
start-region guidance for mission planning.

%% ====================================================================
\section{Problem Formulation}
%% ====================================================================

\subsection{Frames and Kinematics}
Two frames are used:
\begin{itemize}[leftmargin=1.4em]
  \item LVLH frame $\{L\}$ for relative-state propagation,
  \item target body frame $\{B\}$ for docking geometry and LOS constraints.
\end{itemize}
The target body rotates about $z$ with angle $\theta(t)=\omega_t t$.
Transformations between frames include the rotational transport term
on velocity:
\[
  r_B = R(-\theta)\,r_L, \qquad
  v_B = R(-\theta)\,v_L - \omega \times r_B.
\]

\subsection{Relative Dynamics}
A 3D double-integrator model is used in LVLH with bounded acceleration:
\[
  x_{k+1} = A_d\,x_k + B_d\,u_k, \qquad \|u_k\| \le a_{\max},
\]
where $x = [r^T, v^T]^T \in \mathbb{R}^6$, with sample time $T_s = 0.1$~s.

\subsection{LOS Corridor Constraints}
For state $x_B$ in target body frame, the LOS corridor is enforced as:
\[
  A_c\,x_B \le b_c,
\]
with
\[
A_c =
\begin{bmatrix}
  0 & -1 & 0 & 0 & 0 & 0 \\
  c_x & -1 & 0 & 0 & 0 & 0 \\
  -c_x & -1 & 0 & 0 & 0 & 0 \\
  0 & -1 & c_z & 0 & 0 & 0 \\
  0 & -1 & -c_z & 0 & 0 & 0
\end{bmatrix},\quad
b_c =
\begin{bmatrix}
  -r_h \\
  x_0 - c_x r_h \\
  x_0 - c_x r_h \\
  z_0 - c_z r_h \\
  z_0 - c_z r_h
\end{bmatrix},
\]
where $r_h = 0.50$~m is the hold radius.
Because the target tumbles, the LOS corridor \emph{rotates with it}: a state
that is inside the cone at $t=0$ may leave the cone at $t > 0$ if the chaser
cannot track the rotation.  This is the key difficulty.

%% ====================================================================
\section{Guidance and Control Architecture}
%% ====================================================================

A two-layer controller is used:
\begin{enumerate}[leftmargin=1.6em]
  \item \textbf{Spiral reference generation}: a smooth approach profile is
    generated in $\{B\}$ from range $\approx 195$~m down to $0.5$~m along
    the $+y_B$ docking axis, emphasising lateral (${x_B, z_B}$) nulling
    before axial closure.

  \item \textbf{QP-based safety filter}: at each step a small QP is solved:
    \[
      \min_u \; \|u - u_{\rm nom}\|^2 + w_u\,\|u\|^2
                + w_{\Delta u}\,\|u - u_{k-1}\|^2
    \]
    subject to the LOS, radial, keepout, and velocity constraints propagated
    over a short prediction horizon ($N = 6$ steps).
\end{enumerate}

\paragraph{Input-rate penalty.}
The term $w_{\Delta u}\,\|u_k - u_{k-1}\|^2$ is \emph{mandatory} for
physically plausible thrust commands.  Without it, the safety projection
produces chattering control with $>$400 sign changes per run; with it,
sign changes drop to $\sim$30--40 and the commanded acceleration profile
is smooth.

\paragraph{Body-frame tracking weights.}
Lateral axes ($x_B, z_B$) receive high tracking weight to drive the chaser
onto the docking centerline; the axial axis ($y_B$) receives a low weight
to prevent overshoot during range closure.

\paragraph{Hold mode.}
Once the chaser reaches the hold band $[0.45, 0.55]$~m with low speed,
the controller switches to station-keeping that tracks a fixed docking
direction in $\{B\}$, rotating with the target.

%% ====================================================================
\section{Reachability-Based Safe Start-Region Analysis}
\label{sec:reachability}
%% ====================================================================

\subsection{Motivation}
The central question is: \emph{given an initial position $(x_B, y_B)$ inside
the LOS cone with $v_0=0$, can the chaser remain inside the rotating cone
for the entire approach?}  This depends on:
\begin{itemize}[leftmargin=1.4em]
  \item the tumble rate $\omega_t$ (faster rotation $\Rightarrow$ harder),
  \item the thrust authority $a_{\max}$ (more thrust $\Rightarrow$ easier).
\end{itemize}

\subsection{Method: Directional Per-Constraint Erosion}

The analysis is purely geometric---no closed-loop simulation is required.

\paragraph{Step 1: Per-constraint static margin.}
For each grid point $(x_B, y_B)$ with $z_B = 0$ inside the cone, compute the
per-constraint half-space margin:
\[
  s_i(x_B, y_B) = b_i - a_i^T [x_B, y_B, 0]^T, \qquad i = 1,\dots,5.
\]

\paragraph{Step 2: Directional erosion from rotation transport.}
When the body frame rotates at $\omega_t$ about $z$, an inertially-stationary
chaser at $(x_B, y_B)$ sees an apparent body-frame velocity
\[
  v_{\rm rot} = [\,\omega_t\, y_B,\; -\omega_t\, x_B,\; 0\,]^T,
\]
arising from the inverse rotation transport ($-\omega \times r_B$).
The rate of change of each constraint's slack is
\[
  \dot s_i = -a_i^T\, v_{\rm rot}.
\]
If $\dot s_i < 0$ (margin shrinking), the controller must apply
$a_{\max}$ to arrest the drift.  Settling-time analysis gives
\[
  \delta_i = \frac{1}{2}\,\frac{\dot s_i^2}{a_{\max}}, \qquad
  \text{if } \dot s_i < 0; \quad \delta_i = 0 \text{ otherwise.}
\]
A position is classified as ``safe'' if $s_i > \delta_i$ for all $i$.

\paragraph{Directional asymmetry.}
Because $v_{\rm rot}$ depends on both the \emph{sign} of $\omega_t$ and
the \emph{sign} of $x_B$, the erosion is \textbf{asymmetric in $x_B$}.
For clockwise target rotation ($\omega_t < 0$), the left-hand cone
boundary erodes while the right-hand boundary gains margin, tilting
the safe region toward $+x_B$: a chaser starting on the ``upstream''
side of the rotation has more margin because the cone sweeps
\emph{away} from it initially.  The opposite holds for counter-clockwise
rotation.  This matches the physical intuition that rotation direction
breaks the symmetry of the approach corridor.

\paragraph{Step 3: Overlay visualisation.}
For each tumble rate, a single overlay plot shows:
\begin{enumerate}[leftmargin=1.4em]
  \item the initial LOS cone boundary,
  \item nested safe regions for $a_{\max} \in \{0.20, 0.10, 0.05, 0.01\}$~m/s$^2$,
    layered largest to smallest.
\end{enumerate}
The safe regions remain monotonically nested in $a_{\max}$, but are no
longer symmetric in $x_B$.

\subsection{Scope and Honesty}
This method is labelled a \emph{conservative inner approximation}, not a
formal proof.  The per-constraint erosion provides a conservative margin but
does not constitute a certified invariant set in the control-theoretic sense.
A full polyhedral backward-reachable set computation (e.g., via pre-set
iteration for the time-varying constraint polytope) is left as planned
future work.

%% ====================================================================
\section{Scenario Sweep and Figure Plan}
\label{sec:figures}
%% ====================================================================

The study sweeps:
\begin{itemize}[leftmargin=1.4em]
  \item target tumble rates: $\omega_t \in \{1, 2, 3, 4, 5\}$~deg/s
    (5 scenarios),
  \item thrust-to-mass ratios: $a_{\max} \in \{0.01, 0.05, 0.10, 0.20\}$~m/s$^2$
    (4 chaser cases).
\end{itemize}

\paragraph{Figure plan.}
\begin{enumerate}[leftmargin=1.4em]
  \item \textbf{Per-tumble overlay maps} ($5\times 1$ figures):
    \texttt{safe\_maps\_v0\_zero\_omega\_\{1,2,3,4,5\}deg.pdf}.
    Each shows the LOS cone boundary and overlaid safe regions for all four
    thrust levels, layered from $0.20$ (largest, lightest) to $0.01$
    (smallest, darkest).

  \item \textbf{Combined overview}: \texttt{safe\_maps\_v0\_zero.pdf},
    a panel figure with one subplot per tumble rate.

  \item \textbf{Single-trajectory plots} (dev mode generates these as both
    PNG and PDF):
    \begin{itemize}[leftmargin=1em]
      \item \texttt{body\_position\_vs\_time}: chaser relative position in $\{B\}$,
      \item \texttt{constraint\_margin\_vs\_time}: min LOS margin $\min(b - Ax)$,
      \item \texttt{control\_and\_dv\_proxy}: control magnitude + cumulative $\Delta v$,
      \item \texttt{control\_components\_vs\_time}: per-axis commanded acceleration,
      \item \texttt{mpc\_cost\_vs\_time}: MPC cost (log + clipped-linear dual panel),
      \item \texttt{mpc\_cost\_breakdown}: per-term breakdown (position, velocity,
        input, rate),
      \item \texttt{range\_vs\_time}: range with monotonic markers and hold switch,
      \item \texttt{state\_components\_body}: full state in $\{B\}$,
      \item \texttt{trajectory\_body\_3d}: 3D trajectory in $\{B\}$ with LOS corridor.
    \end{itemize}
\end{enumerate}

%% ====================================================================
\section{Comparison with Industry Practice}
%% ====================================================================

In current industrial RPOD mission design (e.g., ESA ClearSpace-1, MEV),
safety verification of the approach corridor is performed via large-scale
Monte Carlo campaigns that sample initial conditions, disturbances, and
navigation errors.  While effective for coverage assessment, Monte Carlo
does not produce a geometric characterisation of the safe start set
or its dependence on system parameters.

The reachability-based maps proposed here complement Monte Carlo by
providing:
\begin{itemize}[leftmargin=1.4em]
  \item a \emph{structured} view of how the safe set shrinks with tumble rate
    and grows with thrust authority,
  \item an \emph{inner approximation} that is conservative by construction
    (erosion margin accounts for settling-time drift),
  \item a fast screening tool for mission-level trade studies on thruster
    sizing and tumble rate constraints.
\end{itemize}

%% ====================================================================
\section{Results (Planned)}
%% ====================================================================

The following results are generated by the accompanying code:
\begin{itemize}[leftmargin=1.4em]
  \item single-trajectory demonstration with smooth control
    (input rate penalty $w_{\Delta u} = 12$),
  \item feasibility fraction table for the 5$\times$4 tumble/thrust grid,
  \item per-tumble overlay feasibility maps.
\end{itemize}

Performance tables and figures are auto-generated and saved in
\texttt{IAC/data/} and \texttt{IAC/figures/}.

%% ====================================================================
\section{Conclusion}
%% ====================================================================

We presented a reachability-inspired method for computing safe start-region
feasibility maps under rotating LOS constraints caused by target tumbling.
The method uses a purely geometric, directional per-constraint erosion
analysis: for each half-space of the LOS corridor, the rotation-induced
apparent velocity $v_{\rm rot} = [\omega\,y_B,\; -\omega\,x_B,\; 0]^T$
determines whether that constraint is eroding or gaining margin.
Only eroding constraints contribute settling-time drift
$\delta_i = \frac{1}{2}\dot{s}_i^2/a_{\max}$.
This yields asymmetric, nested inner approximations of the persistently
feasible set---correctly capturing the physical asymmetry introduced by the
rotation direction---and provides actionable guidance for mission planners
beyond what Monte Carlo alone can offer.

Planned future work includes:
\begin{itemize}[leftmargin=1.4em]
  \item formal backward-reachable set computation via polyhedral pre-set
    iteration for the time-varying constraint polytope,
  \item extension to non-zero initial velocity ($v_0 \ne 0$),
  \item robustness margins for navigation uncertainty and unmodelled dynamics.
\end{itemize}

\end{document}
