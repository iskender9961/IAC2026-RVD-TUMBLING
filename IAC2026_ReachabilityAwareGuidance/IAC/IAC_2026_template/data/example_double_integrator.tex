\subsection{Illustrative Reachability Example: Double Integrator}
\label{sec:example_di}

To illustrate the reachability concepts before applying them to the CWH dynamics, we consider a 2D double integrator:
\begin{equation}
\mathbf{x}_{k+1} = \underbrace{\begin{bmatrix} 1 & 1 \\ 0 & 1 \end{bmatrix}}_{A}\,\mathbf{x}_k + \underbrace{\begin{bmatrix} 0.5 \\ 1 \end{bmatrix}}_{B}\,u_k + \mathbf{w}_k,
\end{equation}
with state constraints $|x_1| \le 5$, $|x_2| \le 3$ and input bound $|u| \le 1$.

Fig.~\ref{fig:example_overlay} shows the backward reachable sets (safe-start regions) for $N=5$ steps to a target set near the origin, computed under three assumptions:
\begin{itemize}
\item \textbf{Nominal} (green): no disturbance, deterministic guarantee.
\item \textbf{Stochastic} (blue): Gaussian noise $\mathbf{w}_k \sim \mathcal{N}(0, W)$, chance constraints with $\alpha=0.05$.
\item \textbf{Robust} (purple): bounded disturbance $\|\mathbf{w}_k\|_\infty \le w_{\max}$, worst-case guarantee.
\end{itemize}

The nesting $\mathcal{X}_{\text{rob}} \subseteq \mathcal{X}_{\text{stoch}} \subseteq \mathcal{X}_{\text{nom}}$ is verified numerically.  This same hierarchy, applied to the CWH dynamics with rotating LOS constraints, yields the feasibility certification results in Section~\ref{sec:hierarchy}.

\begin{figure}[H]
\centering
\includegraphics[width=0.85\columnwidth]{figures/overlay_comparison.pdf}
\caption{Nested backward reachable sets for a 2D double integrator ($N=5$).  The robust set (purple) is the smallest but provides the strongest guarantee; the nominal set (green) is the largest but assumes zero disturbance.}
\label{fig:example_overlay}
\end{figure}
