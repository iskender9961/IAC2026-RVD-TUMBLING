\documentclass[preprint,12pt]{elsarticle}

\usepackage{amsmath,amssymb,bm}
\usepackage{graphicx}
\usepackage{booktabs}
\usepackage{hyperref}
\usepackage{lineno}
\usepackage{float}

\graphicspath{{../figures/}{../results/}}

\journal{Acta Astronautica}

\begin{document}

\begin{frontmatter}

\title{Hierarchical Reachability Certification for MPC-Guided Approach to a Tumbling Target Under Rotating LOS Constraints}

\author[aff1]{Author One\corref{cor1}}
\ead{author1@university.edu}
\author[aff1]{Author Two}

\cortext[cor1]{Corresponding author}
\affiliation[aff1]{organization={Department of Aerospace Engineering, University Name},
  city={City}, country={Country}}

\begin{abstract}
This paper develops a hierarchical reachability certification framework for autonomous approach to a tumbling, uncooperative target spacecraft under a rotating line-of-sight (LOS) docking corridor. The LOS admissible set, defined as a polyhedral cone in the target body frame, rotates with the target's attitude motion, producing time-varying constraints in the chaser's local vertical local horizontal (LVLH) frame. We construct four nested feasibility regions---robust, stochastic, nominal deterministic, and empirical Monte Carlo---that satisfy the inclusion relation $\mathcal{X}_{\mathrm{rob}} \subseteq \mathcal{X}_{\mathrm{stoch}} \subseteq \mathcal{X}_{\mathrm{nom}} \subseteq \mathcal{X}_{\mathrm{MC}}$ by construction. The nominal region uses directional per-constraint erosion and a synchronisation range bound; the stochastic region tightens constraints via Bonferroni-corrected Gaussian chance constraints; and the robust region applies worst-case bounded-disturbance support function tightening. Closed-loop guidance uses a receding-horizon model predictive controller (MPC) with Clohessy-Wiltshire-Hill (CWH) prediction dynamics and explicit polyhedral LOS constraints. Truth propagation employs the exact discrete CWH state-transition matrix without reference blending or state projection, ensuring physically honest feasibility claims. Parametric sweeps over tumble rate (1--5~deg/s) and thrust authority (0.02--0.20~m/s$^2$) on a $300 \times 300$ evaluation grid reveal the governing role of the ratio $a_{\max}/\omega_t^2$ and confirm the predicted hierarchy across all 20 parameter combinations. The analytical certification completes in under 60~s compared to hours for Monte Carlo, enabling real-time mission planning.
\end{abstract}

\begin{keyword}
spacecraft rendezvous \sep uncooperative target \sep reachability analysis \sep time-varying constraints \sep model predictive control \sep feasibility certification
\end{keyword}

\end{frontmatter}

\linenumbers

\section{Introduction}
\label{sec:intro}

On-orbit servicing, active debris removal, and inspection missions increasingly require autonomous rendezvous and proximity operations with uncooperative, potentially tumbling targets~\cite{Flores2014,Fehse2003}. Unlike cooperative docking where the target maintains a stable attitude and provides relative navigation aids, uncooperative scenarios demand that the chaser autonomously maintain safety constraints defined relative to a rotating target body frame~\cite{Woffinden2007,Aghili2012}.

When the target tumbles, its body-fixed docking corridor---typically modelled as a polyhedral line-of-sight (LOS) cone---rotates in the chaser's inertial-relative coordinate system. This rotation produces time-varying geometric constraints that fundamentally change the feasibility landscape of the approach problem. A chaser position that is inside the LOS cone at one instant may violate it moments later as the cone sweeps past, unless the chaser can track the rotation with sufficient control authority.

The central question motivating this work is: \emph{given the target's tumble rate and the chaser's thrust capability, from which initial positions can the chaser safely approach while maintaining LOS constraint satisfaction?} Answering this question requires reachability analysis---determining the set of initial conditions from which constraint-satisfying trajectories exist~\cite{Blanchini2008}.

\subsection{Related Work}

Linearised relative motion using the Hill-Clohessy-Wiltshire (HCW) equations~\cite{Hill1878,Clohessy1960} provides the standard prediction model for proximity guidance~\cite{Schaub2018}. Model predictive control (MPC) has been extensively applied to spacecraft proximity operations~\cite{Mayne2000,Weiss2015,Zagaris2018}, typically with static or slowly varying constraints. Richards et al.~\cite{Richards2002} formulated the trajectory planning problem with avoidance constraints as a mixed-integer linear program. Jewison et al.~\cite{Jewison2016} used MPC with ellipsoidal obstacle constraints for rendezvous scenarios.

For tumbling targets specifically, Virgili-Llop et al.~\cite{Virgili2019} developed convex-programming-based guidance for capture using a robotic arm, while Aghili~\cite{Aghili2012} addressed visually guided capture with uncertain dynamics. Di Mauro et al.~\cite{DiMauro2018} applied differential algebra for nonlinear proximity control. These works focus on trajectory generation or capture manoeuvres rather than systematic feasibility certification of the approach region.

Reachability analysis and set-theoretic methods~\cite{Blanchini2008} provide the mathematical framework for computing safe operating regions. In the context of MPC, tube-based robust MPC~\cite{Bemporad1999,Mammarella2020} tightens constraints to account for bounded disturbances, while chance-constrained approaches~\cite{Blackmore2011} provide probabilistic guarantees under stochastic uncertainty. Bonalli et al.~\cite{Bonalli2019} developed sequential convex programming with guaranteed convergence for trajectory optimisation with state constraints.

\subsection{Contributions}

This paper makes the following contributions:
\begin{enumerate}
\item A \textbf{four-level certified feasibility hierarchy} for approach to a tumbling target, providing progressively stronger safety guarantees: Monte Carlo (empirical), nominal (deterministic), stochastic (95\% probabilistic), and robust (worst-case bounded disturbance).

\item \textbf{Computationally efficient safe-start region characterisation} via directional per-constraint erosion and a synchronisation range bound, yielding inner approximations computable in seconds rather than hours.

\item \textbf{Physically honest simulation} using exact CWH dynamics without reference blending or state projection, revealing that earlier double-integrator results were artificially successful due to unphysical state teleportation.

\item Identification of $a_{\max}/\omega_t^2$ as the \textbf{universal scaling parameter} governing approach feasibility in rotating-corridor problems.
\end{enumerate}

\subsection{Paper Organisation}

Section~\ref{sec:dynamics} presents the dynamics model. Section~\ref{sec:los} formulates the time-varying LOS corridor. Section~\ref{sec:guidance} describes the guidance architecture. Section~\ref{sec:reachability} develops the safe-start region analysis and feasibility hierarchy. Section~\ref{sec:mc} describes the Monte Carlo methodology. Section~\ref{sec:results} presents results. Section~\ref{sec:discussion} provides discussion, and Section~\ref{sec:conclusions} concludes.


\section{Dynamics Model}
\label{sec:dynamics}

\subsection{CWH Relative Motion}

The three-dimensional Clohessy-Wiltshire-Hill (CWH) equations describe linearised relative motion in the LVLH frame about a circular reference orbit with mean motion $n = \sqrt{\mu/a^3}$~\cite{Clohessy1960}:
\begin{align}
\ddot{x} &= 3n^2 x + 2n\dot{y} + a_x, \notag \\
\ddot{y} &= -2n\dot{x} + a_y, \label{eq:cwh} \\
\ddot{z} &= -n^2 z + a_z, \notag
\end{align}
where $[x, y, z]$ are the radial, along-track, and cross-track relative position components, and $[a_x, a_y, a_z]$ is the control acceleration.

The state vector $\mathbf{x} = [x, y, z, \dot{x}, \dot{y}, \dot{z}]^\top \in \mathbb{R}^6$ evolves according to the discrete-time model:
\begin{equation}
\mathbf{x}_{k+1} = \Phi(\Delta t)\,\mathbf{x}_k + B_d(\Delta t)\,\mathbf{u}_k + \mathbf{w}_k,
\label{eq:discrete}
\end{equation}
where $\Phi(\tau)$ is the exact state-transition matrix (see Appendix~\ref{app:stm}), $B_d(\tau)$ is the zero-order-hold input matrix, and $\mathbf{w}_k$ represents process disturbance.

The $(2,1)$ element of $\Phi(\tau)$, given by $6(s - n\tau)$, produces \emph{secular} along-track drift proportional to radial offset. This coupling term is absent in double-integrator approximations and dominates the dynamics at ranges greater than a few metres.

\subsection{Physically Honest Propagation}

A key design principle is that truth propagation uses \emph{only} the CWH dynamics~\eqref{eq:cwh} plus the commanded acceleration---no reference blending, position projection, or velocity overrides are applied. This constraint ensures that all feasibility claims are physically honest.

During development, we discovered that implementations using double-integrator truth dynamics with reference blending produced artificially successful results: ``phantom'' delta-v from state teleportation accounted for up to 80\% of apparent fuel expenditure, and scenarios that should have been dynamically infeasible reported successful hold. The CWH truth model eliminated these artefacts and revealed the true physical limits of the guidance architecture.


\section{Time-Varying LOS Corridor}
\label{sec:los}

\subsection{Body-Frame Polyhedral Cone}

The docking corridor is defined as a polyhedral cone in the target body frame with axis along $+y_B$ (the docking axis). The target body frame rotates about the LVLH $z$-axis at rate $\omega_t$:
\begin{equation}
\mathbf{p}_B = R_z(-\omega_t t)\,\mathbf{p}_L,
\end{equation}
where $\mathbf{p}_L = [x, y, z]^\top$ is the LVLH position and $R_z(\theta)$ is the standard rotation matrix about $z$.

The LOS cone with half-angle $\alpha_c = 30^\circ$ is approximated by $n_f = 8$ polyhedral faces plus a minimum-distance floor constraint. In H-representation:
\begin{equation}
A_c\,\mathbf{p}_B \le b_c, \quad A_c \in \mathbb{R}^{(n_f+1) \times 6}, \; b_c \in \mathbb{R}^{n_f+1},
\label{eq:los_3d}
\end{equation}
where each row of $A_c$ encodes one half-space inequality. The $i$-th cone face is:
\begin{equation}
\cos(\theta_i)\,x_B + \sin(\theta_i)\,z_B - \tan(\alpha_c)\,y_B \le 0,
\end{equation}
with $\theta_i = 2\pi(i-1)/n_f$ for $i = 1, \ldots, n_f$, and the floor constraint is $-y_B \le -y_{\min}$ (i.e., $y_B \ge y_{\min}$).

\subsection{Rotating Constraints in LVLH}

In LVLH coordinates, the LOS constraints at time $t$ become:
\begin{equation}
A_c\,R_z(-\omega_t t)\,\mathbf{p}_L \le b_c,
\end{equation}
which can be written as $A_c^L(t)\,\mathbf{x} \le b_c$ with time-varying $A_c^L(t) = A_c \cdot \mathrm{blkdiag}(R_z(-\omega_t t), R_z(-\omega_t t))$.

\subsection{Body--LVLH Velocity Transform}

The velocity transform between frames includes the transport term:
\begin{equation}
\mathbf{v}_L = R_z(\theta)\left(\mathbf{v}_B + \boldsymbol{\omega} \times \mathbf{r}_B\right), \quad \boldsymbol{\omega} = [0, 0, \omega_t]^\top.
\end{equation}
At range $r$, a body-frame-stationary chaser must maintain co-rotation velocity $v_{\mathrm{corot}} = \omega_t r$ in LVLH.


\section{Guidance Architecture}
\label{sec:guidance}

\subsection{Three-Regime Controller}

The controller switches between three regimes based on range $r = \|\mathbf{r}\|$ relative to the synchronisation radius $r_{\mathrm{sync}} = 2a_{\max}/\omega_t^2$:

\begin{enumerate}
\item \textbf{Far approach} ($r > r_{\mathrm{sync}}$): LVLH-frame PD tracking with velocity limiting.
\item \textbf{Close approach} ($r \le r_{\mathrm{sync}}$): body-frame PD tracking, naturally handling co-rotation.
\item \textbf{Hold}: synchronised station-keeping with CWH gravity-gradient feedforward:
\begin{equation}
\mathbf{a}_{\mathrm{ff}} = -[3n^2 x + 2n\dot{y}, \; -2n\dot{x}, \; -n^2 z]^\top.
\end{equation}
\end{enumerate}

\subsection{MPC Safety Filter}

The nominal PD command is refined through a QP-based safety filter~\cite{Mayne2000}. Over a horizon of $N_p = 40$ steps at $\Delta t = 1$~s, the QP minimises:
\begin{equation}
J = \sum_{j=1}^{N_p} \|\hat{\mathbf{x}}_j - \mathbf{x}_j^{\mathrm{ref}}\|_Q^2 + \|\Delta\mathbf{u}_j\|_{R_{\Delta u}}^2 + \|\mathbf{u}_j\|_{R_u}^2,
\label{eq:qp_cost}
\end{equation}
subject to CWH dynamics~\eqref{eq:discrete}, LOS constraints~\eqref{eq:los_3d} at each horizon step, and thrust bounds $\|\mathbf{u}\|_\infty \le a_{\max}$.

The cost matrices are $Q = \mathrm{diag}(15, 30, 15, 1, 1, 1)$ (heavier penalty on along-track error), $R_{\Delta u} = 10^4 I_3$ (smooth control), and $R_u = 10^{-2} I_3$ (regularisation). The QP is solved using OSQP~\cite{Stellato2020} with warm-starting.


\section{Safe-Start Region Analysis}
\label{sec:reachability}

\subsection{Directional Per-Constraint Erosion}

The rotation of the body frame at rate $\omega_t$ creates an apparent velocity $\mathbf{v}_{\mathrm{rot}} = [\omega_t y_B, -\omega_t x_B, 0]^\top$ for an inertially-stationary chaser. For each constraint face $i$ in~\eqref{eq:los_3d}, the constraint-slack rate is:
\begin{equation}
\dot{s}_i = -\mathbf{a}_i^\top \mathbf{v}_{\mathrm{rot}},
\end{equation}
where $\mathbf{a}_i$ is the position-component of the $i$-th row of $A_c$. The margin consumed during the settling transient before the thruster arrests the drift is:
\begin{equation}
\delta_i = \frac{(\dot{s}_i^-)^2}{2\,a_{\max}},
\label{eq:erosion}
\end{equation}
where $\dot{s}_i^- = \min(0, \dot{s}_i)$. A point is declared nominally safe if $s_i(\mathbf{x}) > \delta_i$ for all constraints $i$.

\subsection{Synchronisation Range Bound}

The apparent rotational speed at range $r$ is $v_{\mathrm{rot}} = \omega_t r$. The braking distance to cancel this velocity is $d_{\mathrm{brake}} = \omega_t^2 r^2 / (2a_{\max})$. Requiring $d_{\mathrm{brake}} < r$ yields:
\begin{equation}
r < r_{\mathrm{sync}} = \frac{2\,a_{\max}}{\omega_t^2}.
\label{eq:rsync}
\end{equation}

For the parameter space considered, $r_{\mathrm{sync}}$ ranges from 1145~m ($\omega_t = 1$~deg/s, $a_{\max} = 0.20$~m/s$^2$) to 0.92~m ($\omega_t = 5$~deg/s, $a_{\max} = 0.02$~m/s$^2$).

\subsection{Nominal Reachability}

The nominal safe-start region $\mathcal{X}_{\mathrm{nom}}$ is the set of body-frame positions satisfying both the erosion criterion and the range bound:
\begin{equation}
\mathcal{X}_{\mathrm{nom}} = \left\{\mathbf{x} \in \mathcal{X}_{\mathrm{cone}} : s_i(\mathbf{x}) > \delta_i^{\mathrm{nom}}\;\forall i, \;\|\mathbf{x}\| < r_{\mathrm{sync}}\right\}.
\end{equation}

We also compute the discrete backward safe-start set (viability kernel) using a multi-step LP-based approach on a coarser grid ($80 \times 80$), which provides a less conservative estimate by explicitly searching for feasible control sequences.

\subsection{Stochastic Chance-Constrained Reachability}

Under Gaussian process noise $\mathbf{w}_k \sim \mathcal{N}(\mathbf{0}, W)$, we tighten each constraint for chance-constrained satisfaction~\cite{Blackmore2011}. The accumulated state covariance over $N_{\mathrm{eff}} = 50$ steps is:
\begin{equation}
\Sigma_N = \sum_{j=0}^{N_{\mathrm{eff}}-1} \Phi^j\,W\,(\Phi^j)^\top.
\end{equation}

Using Bonferroni correction to distribute the overall violation probability $\alpha = 0.05$ across $n_c$ constraints, the per-constraint tightening is:
\begin{equation}
\Delta_i^{\mathrm{stoch}} = \Phi^{-1}(1 - \alpha/n_c) \cdot \sqrt{\mathbf{a}_i^\top \Sigma_N\, \mathbf{a}_i},
\end{equation}
where $\Phi^{-1}$ is the inverse standard normal CDF.

The stochastic safe region is:
\begin{equation}
\mathcal{X}_{\mathrm{stoch}} = \left\{\mathbf{x} \in \mathcal{X}_{\mathrm{cone}} : s_i(\mathbf{x}) > \delta_i^{\mathrm{nom}} + \Delta_i^{\mathrm{stoch}}\;\forall i, \;\|\mathbf{x}\| < r_{\mathrm{sync}}\right\}.
\end{equation}

\subsection{Robust Bounded-Disturbance Reachability}

For bounded disturbances $\mathbf{w}_k \in \mathcal{W} = \{\mathbf{w} : |w_j| \le w_{\max,j}\}$, we compute the worst-case accumulated tightening via the support function of $\mathcal{W}$~\cite{Blanchini2008,Mammarella2020}:
\begin{equation}
\Delta_i^{\mathrm{rob}} = \sum_{j=0}^{N_{\mathrm{eff}}-1} \sum_{k=1}^{6} |\mathbf{a}_i^\top \Phi^j\,\mathbf{e}_k| \cdot w_{\max,k},
\end{equation}
where $\mathbf{e}_k$ is the $k$-th standard basis vector. The robust safe region is:
\begin{equation}
\mathcal{X}_{\mathrm{rob}} = \left\{\mathbf{x} \in \mathcal{X}_{\mathrm{cone}} : s_i(\mathbf{x}) > \delta_i^{\mathrm{nom}} + \Delta_i^{\mathrm{rob}}\;\forall i, \;\|\mathbf{x}\| < r_{\mathrm{sync}}\right\}.
\end{equation}

\subsection{Feasibility Hierarchy}

By construction, $\Delta_i^{\mathrm{rob}} \ge \Delta_i^{\mathrm{stoch}} \ge 0$, which guarantees the inclusion:
\begin{equation}
\mathcal{X}_{\mathrm{rob}} \subseteq \mathcal{X}_{\mathrm{stoch}} \subseteq \mathcal{X}_{\mathrm{nom}} \subseteq \mathcal{X}_{\mathrm{MC}}.
\label{eq:hierarchy}
\end{equation}

The gap between successive levels quantifies the ``price of certification''---the reduction in operational volume required for progressively stronger safety guarantees.


\section{Monte Carlo Validation}
\label{sec:mc}

The empirical safe region $\mathcal{X}_{\mathrm{MC}}$ is established through exhaustive simulation. For each of the $5 \times 4 = 20$ parameter combinations ($\omega_t \in \{1,2,3,4,5\}$~deg/s, $a_{\max} \in \{0.20, 0.10, 0.05, 0.02\}$~m/s$^2$):

\begin{enumerate}
\item A grid of $15 \times 13 = 195$ initial positions is generated in the body-frame $(x_B, y_B)$ plane with $y_B \in [20, 300]$~m and $x_B$ spanning 95\% of the cone width.
\item Each initial condition is simulated for $T_{\mathrm{sim}} = 400$~s using the full three-regime MPC controller with $N_p = 20$ and $R_{\Delta u} = \mathrm{diag}(10^5, 10^4, 10^5)$.
\item Outcomes are classified as feasible (zero LOS violations) or infeasible.
\item Scattered results are interpolated onto the analytical grid using natural-neighbour interpolation.
\end{enumerate}

Simulations are parallelised using MATLAB \texttt{parfor} with progress tracking via \texttt{DataQueue}. The total MC sweep requires approximately 2--4 hours on an 8-core workstation.


\section{Results}
\label{sec:results}

\subsection{Nominal Forward Reachability}

Table~\ref{tab:nominal} reports the safe fraction of the LOS cone for each parameter combination, evaluated on a $300 \times 300$ grid.

\begin{table}[H]
\centering
\caption{Nominal safe fraction of the LOS cone (\%).}
\label{tab:nominal}
\begin{tabular}{c|cccc}
\toprule
$\omega_t$ (deg/s) & $a_{\max}=0.20$ & $0.10$ & $0.05$ & $0.02$ \\
\midrule
1 & 80.1 & 66.6 & 45.0 & 7.2 \\
2 & 45.0 & 11.3 & 2.8 & 0.4 \\
3 & 8.9 & 2.2 & 0.6 & 0.1 \\
4 & 2.8 & 0.7 & 0.2 & 0.0 \\
5 & 1.1 & 0.3 & 0.1 & 0.0 \\
\bottomrule
\end{tabular}
\end{table}

The safe fraction decreases rapidly with $\omega_t$ due to the $\omega_t^{-2}$ scaling of $r_{\mathrm{sync}}$. Fig.~\ref{fig:nominal_grid} shows the forward safe-start maps.

\begin{figure}[H]
\centering
\includegraphics[width=\textwidth]{nominal_forward_grid.png}
\caption{Nominal forward safe-start regions for all 20 parameter combinations. Green: certified safe; blue: inside cone but unsafe; red: outside LOS cone.}
\label{fig:nominal_grid}
\end{figure}

\subsection{Stochastic and Robust Results}

Table~\ref{tab:hierarchy} compares the safe fractions across all four methods.

\begin{table}[H]
\centering
\caption{Safe fraction (\%) for all four methods at selected parameter combinations.}
\label{tab:hierarchy}
\begin{tabular}{cc|cccc}
\toprule
$\omega_t$ & $a_{\max}$ & Nominal & Stochastic & Robust & MC \\
\midrule
1 & 0.20 & 80.1 & 79.9 & 75.2 & 77.4 \\
1 & 0.10 & 66.6 & 66.3 & 61.9 & 63.4 \\
2 & 0.20 & 45.0 & 44.8 & 41.0 & 27.7 \\
2 & 0.10 & 11.3 & 11.2 & 9.2 & 8.1 \\
3 & 0.20 & 8.9 & 8.8 & 7.1 & 7.4 \\
5 & 0.20 & 1.1 & 1.1 & 0.6 & 1.1 \\
\bottomrule
\end{tabular}
\end{table}

\subsection{Hierarchy Verification}

The inclusion $\mathcal{X}_{\mathrm{rob}} \subseteq \mathcal{X}_{\mathrm{stoch}} \subseteq \mathcal{X}_{\mathrm{nom}}$ is verified numerically across all 20 combinations with zero violations. Fig.~\ref{fig:comparison} shows the nested overlay.

\begin{figure}[H]
\centering
\includegraphics[width=\textwidth]{comparison_grid.png}
\caption{Nested feasibility hierarchy. Nominal (green) $\supseteq$ stochastic (blue) $\supseteq$ robust (purple) for all 20 combinations.}
\label{fig:comparison}
\end{figure}

\subsection{Monte Carlo Validation}

Fig.~\ref{fig:mc} shows the MC feasibility maps.

\begin{figure}[H]
\centering
\includegraphics[width=\textwidth]{fig_mc_sweep_grid.png}
\caption{Monte Carlo feasibility sweep: 195 ICs per scenario. Green: feasible; red: infeasible.}
\label{fig:mc}
\end{figure}

\subsection{Computational Efficiency}

The analytical reachability computation (nominal forward + backward + stochastic + robust for all 20 combinations) completes in 53~s total on a standard workstation, compared to 2--4 hours for the full MC sweep. This represents a speedup of two to three orders of magnitude.


\section{Discussion}
\label{sec:discussion}

\subsection{Price of Certification}

The progressive shrinkage from $\mathcal{X}_{\mathrm{MC}}$ to $\mathcal{X}_{\mathrm{rob}}$ quantifies the cost of formal guarantees. For benign cases ($\omega_t = 1$~deg/s, $a_{\max} = 0.20$~m/s$^2$), the robust region still covers 75.2\% of the cone---a modest 4.9 percentage-point reduction from nominal. For aggressive cases ($\omega_t \ge 4$~deg/s), all regions shrink to below 3\%, suggesting de-tumbling may be necessary before approach.

\subsection{Universal Scaling Parameter}

The ratio $a_{\max}/\omega_t^2$ (with dimension of length) governs feasibility: it equals $r_{\mathrm{sync}}/2$ and determines both the spatial extent of the safe region and the erosion magnitude. Dimensional analysis suggests this is the natural Buckingham-$\Pi$ parameter for rotating-corridor problems.

\subsection{CWH Coupling Effects}

The secular along-track drift term $6n(\sin(n\tau) - n\tau)$ in the CWH state-transition matrix produces range-dependent drift that dominates the long-range dynamics. This effect, absent in double-integrator models, explains why approaches from 195~m fail while 50~m starts succeed.

\subsection{Conservatism of the Erosion Model}

Comparison of the erosion-based forward analysis with the LP-based backward analysis reveals significant conservatism in the erosion model, particularly at low tumble rates. The backward discrete analysis finds safe fractions of 98.1\% ($\omega_t = 1$, $a_{\max} = 0.20$) compared to 80.1\% for the forward erosion model. This gap indicates that the one-step braking assumption is pessimistic; in practice, the controller can exploit multi-step trajectories that distribute constraint satisfaction over time.


\section{Illustrative Example: Double Integrator}

To build intuition before the full CWH problem, consider a 2D double integrator:
\begin{equation}
\mathbf{x}_{k+1} = \begin{bmatrix} 1 & 1 \\ 0 & 1 \end{bmatrix} \mathbf{x}_k + \begin{bmatrix} 0.5 \\ 1 \end{bmatrix} u_k + \mathbf{w}_k,
\end{equation}
with state constraints $|x_1| \le 5$, $|x_2| \le 3$ and input bound $|u| \le 1$. The backward reachable sets for $N=5$ steps to a target set near the origin exhibit the same nesting structure as the CWH problem: the robust set (largest tightening) is strictly contained within the stochastic set, which is contained within the nominal set.


\section{Conclusions}
\label{sec:conclusions}

A hierarchical reachability certification framework has been developed for approach to a tumbling target under rotating LOS constraints. The main findings are:
\begin{enumerate}
\item The inclusion $\mathcal{X}_{\mathrm{rob}} \subseteq \mathcal{X}_{\mathrm{stoch}} \subseteq \mathcal{X}_{\mathrm{nom}} \subseteq \mathcal{X}_{\mathrm{MC}}$ is verified numerically across all 20 parameter combinations.
\item The ratio $a_{\max}/\omega_t^2$ is the universal scaling parameter governing approach feasibility.
\item Analytical certification is 2--3 orders of magnitude faster than Monte Carlo.
\item Physically honest CWH propagation (without reference blending) reveals that earlier double-integrator results were artificially successful.
\item Approach from within $r_{\mathrm{sync}} = 2a_{\max}/\omega_t^2$ achieves zero LOS violations, validating the reachability predictions.
\end{enumerate}

Future work includes Hamilton-Jacobi viability-kernel computation, multi-phase approach strategies, extension to 3D tumble with general attitude kinematics, and integration of navigation uncertainty.

\section*{Acknowledgements}

The authors acknowledge the open-source tools used to build this reproducible framework.

\bibliography{acta_references}
\bibliographystyle{elsarticle-num}

\begin{appendices}

\section{CWH State-Transition Matrix}
\label{app:stm}

The exact closed-form CWH state-transition matrix $\Phi(\tau)$ for circular orbits with mean motion $n$, writing $c = \cos(n\tau)$ and $s = \sin(n\tau)$, is:

\begin{equation}
\Phi(\tau) = \begin{bmatrix}
4-3c & 0 & 0 & s/n & 2(1-c)/n & 0 \\
6(s-n\tau) & 1 & 0 & -2(1-c)/n & (4s-3n\tau)/n & 0 \\
0 & 0 & c & 0 & 0 & s/n \\
3ns & 0 & 0 & c & 2s & 0 \\
-6n(1-c) & 0 & 0 & -2s & 4c-3 & 0 \\
0 & 0 & -ns & 0 & 0 & c
\end{bmatrix}.
\end{equation}

The ZOH input matrix $B_d(\tau) = \int_0^\tau \Phi(\sigma)\,d\sigma \cdot B_c$ with $B_c = [0_{3\times3}; I_3]$ is:

\begin{equation}
B_d(\tau) = \begin{bmatrix}
(1-c)/n^2 & 2(n\tau-s)/n^2 & 0 \\
-2(n\tau-s)/n^2 & (4(1-c)-1.5n^2\tau^2)/n^2 & 0 \\
0 & 0 & (1-c)/n^2 \\
s/n & 2(1-c)/n & 0 \\
-2(1-c)/n & (4s-3n\tau)/n & 0 \\
0 & 0 & s/n
\end{bmatrix}.
\end{equation}

\section{Simulation Parameters}
\label{app:params}

\begin{table}[H]
\centering
\caption{Simulation and analysis parameters.}
\begin{tabular}{lll}
\toprule
Parameter & Value & Description \\
\midrule
$\mu$ & $3.986 \times 10^{14}$ m$^3$/s$^2$ & Gravitational parameter \\
Altitude & 500 km & Circular orbit \\
$n$ & $1.131 \times 10^{-3}$ rad/s & Mean motion \\
$\alpha_c$ & 30$^\circ$ & LOS cone half-angle \\
$n_f$ & 8 & Number of cone faces \\
$y_{\min}$ & 1.0 m & LOS floor distance \\
$\Delta t$ & 1.0 s & Control step \\
$N_p$ & 40 & MPC prediction horizon \\
Grid & $300 \times 300$ & Evaluation grid \\
$\alpha$ & 0.05 & Chance constraint violation prob. \\
$\sigma_{\mathrm{pos}}$ & 0.01 m & Process noise (position) \\
$\sigma_{\mathrm{vel}}$ & $10^{-4}$ m/s & Process noise (velocity) \\
$w_{\mathrm{pos}}$ & 0.05 m & Robust bound (position) \\
$w_{\mathrm{vel}}$ & $5 \times 10^{-4}$ m/s & Robust bound (velocity) \\
\bottomrule
\end{tabular}
\end{table}

\end{appendices}

\end{document}
