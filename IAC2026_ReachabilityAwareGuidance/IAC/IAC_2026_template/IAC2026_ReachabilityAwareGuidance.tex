\documentclass[]{IAC_style}

\usepackage{amsmath,amssymb}
\usepackage{multirow}

\begin{document}

\IACpaperyear{2026}
\IACpapernumber{IAC-26-C1.8.2-x00001}
\IACconference{77th International Astronautical Congress (IAC 2026)}
\IAClocation{Antalya, T\"{u}rkiye}
\IACdate{5-9 October 2026}

\IACcopyrightA{}

\title{Reachability-Aware Guidance for Approach to a Tumbling Uncooperative Target with Time-Varying LOS Constraints}

\IACauthor{\"{O}mer Burak \.{I}skender$^{\orcidlink{0000-0000-0000-0000}}$}{1}{1}
\IACauthoraffiliation{School of Electrical and Electronic Engineering, Nanyang Technological University, Singapore \normalfont{E-mail:~\authormail{iske0001@e.ntu.edu.sg}}}

\abstract{
This paper presents a reachability-aware guidance architecture for autonomous approach to a tumbling, uncooperative target under a rotating line-of-sight (LOS) docking corridor.
Three nested analytical safe-start sets---robust, stochastic, and nominal---satisfy $\mathcal{X}_{\mathrm{rob}} \subseteq \mathcal{X}_{\mathrm{stoch}} \subseteq \mathcal{X}_{\mathrm{nom}}$ by construction, while an independent Monte~Carlo campaign provides empirical closed-loop validation.
Closed-loop guidance couples a receding-horizon quadratic-program (QP) controller---with state-tracking, input-rate, and terminal penalties---to nonlinear two-body-plus-$J_2$ truth dynamics, ensuring physically honest feasibility claims.
Parametric sweeps over tumble rates 1--5~deg/s and thrust authorities 0.02--0.20~m/s$^2$ on a $300\!\times\!300$ evaluation grid identify $a_{\max}/\omega_t^2$ as the single dimensionless parameter governing approach feasibility, confirm the predicted hierarchy across all 20 parameter combinations with zero point-wise violations, and show that analytical certification completes in 53~s versus 2--4~hours for Monte~Carlo ($\sim$200$\times$ speedup), enabling on-board mission replanning.
}

\IACkeywords{proximity operations, uncooperative target, time-varying LOS corridor, reachability, safe-start region, feasibility hierarchy, MPC, CWH dynamics}

\maketitle
\thispagestyle{fancy}

\section*{Nomenclature}
$n$ \quad mean motion of target orbit (rad/s) \\
$\mathbf{x}=[x,y,z,\dot{x},\dot{y},\dot{z}]^\top$ \quad LVLH relative state (m, m/s) \\
$\mathbf{u}=[a_x,a_y,a_z]^\top$ \quad control acceleration (m/s$^2$) \\
$a_{\max}$ \quad maximum thrust-to-mass ratio (m/s$^2$) \\
$\omega_t$ \quad target tumble rate about body $z$-axis (rad/s) \\
$R_z(\theta)$ \quad rotation matrix about $z$ by angle $\theta$ \\
$\Phi(\tau)$, $B_d(\tau)$ \quad CWH state-transition and input matrices \\
$r_{\mathrm{sync}}$ \quad synchronisation range limit (m) \\
$\delta_i$ \quad directional per-constraint erosion (m) \\
$\mathcal{W}$ \quad bounded disturbance set \\
$\alpha$ \quad chance-constraint violation probability \\

\section*{Acronyms/Abbreviations}
CWH: Clohessy-Wiltshire-Hill; ECI: Earth-centred inertial; LOS: line of sight; LVLH: local vertical local horizontal; MPC: model predictive control; QP: quadratic program; MC: Monte Carlo

%%=========================================================================
\section{Introduction}
%%=========================================================================

Autonomous rendezvous and proximity operations with uncooperative targets are central to on-orbit servicing, active debris removal, and space situational awareness~\cite{Flores2014,Fehse2003,Bonnal2013}. When the target tumbles, its body-fixed docking corridor rotates in the chaser's coordinate frame, producing time-varying geometric constraints whose feasibility depends critically on the interplay between tumble rate and thrust authority~\cite{Virgili2019,Aghili2012}. A chaser position inside the LOS cone at one instant may violate it moments later unless the chaser can co-rotate with sufficient control authority~\cite{Woffinden2007}.

Linearised relative motion using the Hill-Clohessy-Wiltshire (HCW) equations~\cite{Hill1878,Clohessy1960} provides a compact prediction model for proximity guidance~\cite{Schaub2018}. Model predictive control (MPC) with explicit constraint embedding has been widely adopted for safe proximity operations~\cite{Weiss2015,Zagaris2018,Jewison2016,Richards2002}, typically for static keep-out zones or fixed LOS corridors. For tumbling targets, Virgili-Llop et al.~\cite{Virgili2019} developed convex-programming guidance for robotic-arm capture, Di~Mauro et al.~\cite{DiMauro2018} applied differential algebra for nonlinear proximity control, and Grzymisch and Fichter~\cite{Grzymisch2015} derived analytic optimal control for approach to a tumbling target.  These works focus on trajectory generation, not on systematic pre-mission feasibility certification of the approach region.

The central question motivating this work is: \emph{from which initial states can the chaser safely approach and synchronise with the rotating hold point, given its thrust authority and the target's tumble rate?}  Answering this requires computing the safe-start region---the set of initial conditions from which constraint-satisfying trajectories exist~\cite{Blanchini2008,Blanchini1999}.  Set-theoretic methods~\cite{Blanchini2008,Althoff2021} and Hamilton-Jacobi approaches~\cite{Mitchell2005,Bansal2017} provide frameworks for computing safe operating regions.  Tube-based robust MPC~\cite{Mammarella2020,Langson2004,Dong2020,Quartullo2024} tightens constraints against bounded disturbances using mRPI sets; Dong et al.~\cite{Dong2020} specifically addressed tumbling-target rendezvous with output feedback, while Quartullo et al.~\cite{Quartullo2024} introduced variable-horizon tube MPC for rotating targets.  Specht et al.~\cite{Specht2023} developed a full tube-MPC tracking pipeline for free-tumbling debris capture.  Faraci and Lampariello~\cite{Faraci2025} recently addressed the complementary problem of guaranteed reachability---ensuring the chaser can reach all target orientations at terminal time.  Chance-constrained approaches~\cite{Blackmore2011,Mesbah2016} provide probabilistic guarantees.

\textbf{Gap.}  No existing work provides a unified framework that maps the entire approach region into nested feasibility sets for a tumbling target with rotating polyhedral constraints under nominal, stochastic, and robust assumptions simultaneously.  Tube-based MPC embeds robustness within the controller but does not produce pre-mission feasibility maps; reachability-guaranteed methods certify terminal interception but not corridor safety during transit.

\textbf{Contributions.}  This paper makes four contributions:
\begin{enumerate}
\item \textbf{Hierarchical feasibility certification}: three nested analytical safe-start regions satisfying $\mathcal{X}_{\mathrm{rob}} \subseteq \mathcal{X}_{\mathrm{stoch}} \subseteq \mathcal{X}_{\mathrm{nom}}$ by construction, with an independent empirical Monte~Carlo set for closed-loop validation.

\item \textbf{Directional per-constraint erosion with synchronisation bound}: a closed-form inner approximation using the constraint-slack rate and $r_{\mathrm{sync}} = 2a_{\max}/\omega_t^2$.

\item \textbf{Identification of $a_{\max}/\omega_t^2$ as the universal scaling parameter}: all safe-fraction results collapse onto a single curve.

\item \textbf{Physically honest closed-loop validation}: nonlinear two-body-plus-$J_2$ truth dynamics, demonstrating that double-integrator models produce artificially successful approaches.
\end{enumerate}

%%=========================================================================
\section{Mission Scenario}
\label{sec:mission}
%%=========================================================================

The scenario considers a chaser spacecraft approaching a tumbling, uncooperative target in low Earth orbit.  Table~\ref{tab:mission} summarises the parameters.

\begin{table}[H]
\centering
\caption{Mission scenario parameters.}
\label{tab:mission}
\begin{tabular}{lll}
\toprule
\textbf{Parameter} & \textbf{Value} & \textbf{Description} \\
\midrule
\multicolumn{3}{l}{\emph{Orbit}} \\
$\mu$ & $3.986 \times 10^{14}$ m$^3$/s$^2$ & Gravitational parameter \\
Altitude & 500 km & Circular LEO \\
$n$ & $1.131 \times 10^{-3}$ rad/s & Mean motion \\
$J_2$ & $1.083 \times 10^{-3}$ & Zonal harmonic \\
\midrule
\multicolumn{3}{l}{\emph{Target}} \\
$\omega_t$ & $\{1,2,3,4,5\}$ deg/s & Tumble rate about body $z$ \\
\midrule
\multicolumn{3}{l}{\emph{Chaser}} \\
$a_{\max}$ & $\{0.20, 0.10, 0.05, 0.02\}$ m/s$^2$ & Max thrust-to-mass \\
\midrule
\multicolumn{3}{l}{\emph{Docking corridor}} \\
$\alpha_c$ & 30$^\circ$ & LOS cone half-angle \\
$n_f$ & 8 & Polyhedral cone faces \\
$y_{\min}$ & 1.0 m & Corridor floor distance \\
\midrule
\multicolumn{3}{l}{\emph{Simulation}} \\
$T_{\mathrm{sim}}$ & 400--600 s & Duration \\
$\Delta t$ & 1.0 s & Control time step \\
\bottomrule
\end{tabular}
\end{table}


%%=========================================================================
\section{Reference Frames}
\label{sec:frames}
%%=========================================================================

Three coordinate frames are used.

\textbf{Earth-Centred Inertial (ECI) $\mathcal{F}_I$:}
Origin at Earth's centre; $\hat{\mathbf{x}}_I$ toward the vernal equinox, $\hat{\mathbf{z}}_I$ along Earth's spin axis.

\textbf{LVLH $\mathcal{F}_L$:}
Centred on the target:
\begin{equation}
\hat{\mathbf{x}}_L = \frac{\mathbf{r}_t}{\|\mathbf{r}_t\|}, \quad
\hat{\mathbf{z}}_L = \frac{\mathbf{r}_t \times \mathbf{v}_t}{\|\mathbf{r}_t \times \mathbf{v}_t\|}, \quad
\hat{\mathbf{y}}_L = \hat{\mathbf{z}}_L \times \hat{\mathbf{x}}_L,
\label{eq:lvlh_def}
\end{equation}
so $x_L$ is radially outward, $y_L$ approximately along-track, $z_L$ orbit-normal.

\textbf{Target Body $\mathcal{F}_B$:}
Fixed to the tumbling target with $+y_B$ along the docking axis.  Attitude relative to $\mathcal{F}_I$ is tracked by a unit quaternion:
\begin{equation}
\dot{\mathbf{q}}_{IB} = \tfrac{1}{2}\,\mathbf{q}_{IB} \otimes [0;\; \boldsymbol{\omega}_B], \quad \boldsymbol{\omega}_B = [0, 0, \omega_t]^\top.
\label{eq:quat_prop}
\end{equation}


%%=========================================================================
\section{Dynamics Model}
\label{sec:dynamics}
%%=========================================================================

\subsection{Nonlinear Truth Model}

Truth propagation uses full nonlinear dynamics in $\mathcal{F}_I$:
\begin{equation}
\ddot{\mathbf{r}}_I = -\frac{\mu}{\|\mathbf{r}_I\|^3}\,\mathbf{r}_I + \mathbf{a}_{J_2}(\mathbf{r}_I) + \mathbf{a}_{\mathrm{ctrl}},
\label{eq:eci_eom}
\end{equation}
where $\mathbf{a}_{J_2}$ is the standard $J_2$ perturbation acceleration.  Integration uses variable-step Runge-Kutta (\texttt{ode113}) with tolerances $10^{-10}$/$10^{-12}$.  Target and chaser are propagated independently in $\mathcal{F}_I$; relative states are obtained by frame transformation.

\subsection{CWH Prediction Model}

The MPC prediction model uses the CWH equations~\cite{Clohessy1960}:
\begin{align}
\ddot{x} &= 3n^2 x + 2n\dot{y} + a_x, \notag \\
\ddot{y} &= -2n\dot{x} + a_y, \label{eq:cwh} \\
\ddot{z} &= -n^2 z + a_z. \notag
\end{align}

In discrete time with $\mathbf{x}_k = [x, y, z, \dot{x}, \dot{y}, \dot{z}]^\top$:
\begin{equation}
\mathbf{x}_{k+1} = \Phi(\Delta t)\,\mathbf{x}_k + B_d(\Delta t)\,\mathbf{u}_k + \mathbf{w}_k,
\label{eq:discrete}
\end{equation}
where $\Phi(\tau)$ is the exact state-transition matrix with $c = \cos(n\tau)$, $s = \sin(n\tau)$:
\begin{equation}
\Phi(\tau) = \begin{bmatrix}
4 - 3c & 0 & 0 & s/n & 2(1-c)/n & 0 \\
6(s - n\tau) & 1 & 0 & -2(1-c)/n & (4s - 3n\tau)/n & 0 \\
0 & 0 & c & 0 & 0 & s/n \\
3ns & 0 & 0 & c & 2s & 0 \\
-6n(1-c) & 0 & 0 & -2s & 4c-3 & 0 \\
0 & 0 & -ns & 0 & 0 & c
\end{bmatrix}.
\label{eq:phi}
\end{equation}

The $(2,1)$ element $6(s - n\tau)$ produces secular along-track drift proportional to radial offset---a critical coupling absent in double-integrator models.

\subsection{Frame Transformations}

The relative state in $\mathcal{F}_B$ is:
\begin{align}
\mathbf{r}_B &= R_{IB}^\top(\mathbf{r}_c - \mathbf{r}_t), \label{eq:pos_transform} \\
\mathbf{v}_B &= R_{IB}^\top(\mathbf{v}_c - \mathbf{v}_t) - \boldsymbol{\omega}_B \times \mathbf{r}_B, \label{eq:vel_transform}
\end{align}
where the transport term in~\eqref{eq:vel_transform} requires co-rotation velocity $v_{\mathrm{corot}} = \omega_t r$ at range $r$.  This ensures that a body-frame--stationary chaser maintains the correct co-rotation in the inertial frame.

\subsection{Online MPC Linearisation}
\label{sec:linearise}

At every control step $k$, the MPC prediction matrices $(A_{d,k}, B_{d,k})$ are recomputed via forward finite differences ($\epsilon = 10^{-6}$) through the full nonlinear pipeline: body-frame recovery $\to$ ECI propagation (\texttt{ode113}, two-body$+J_2$) $\to$ attitude update $\to$ body-frame projection~\eqref{eq:pos_transform}--\eqref{eq:vel_transform}.  Each column requires one nonlinear propagation ($1 + n_x + n_u = 10$ per step), capturing $J_2$ secular drift and Coriolis coupling that a frozen CWH model would miss.

\paragraph{Model usage distinction.}
The CWH STM $\Phi(\tau)$ is used exclusively for \emph{offline} reachability analysis (Section~\ref{sec:reachability}); all \emph{online} MPC predictions use the finite-difference linearisation.


%%=========================================================================
\section{Time-Varying LOS Corridor}
\label{sec:los}
%%=========================================================================

The docking corridor is a polyhedral cone in $\mathcal{F}_B$ with axis $+y_B$ and half-angle $\alpha_c = 30^\circ$, approximated by $n_f = 8$ half-spaces plus a floor:
\begin{equation}
A_c\,\mathbf{p}_B \le b_c, \quad A_c \in \mathbb{R}^{9 \times 3}.
\label{eq:los_3d}
\end{equation}
The $i$-th face constraint is $\cos(\theta_i)\,x_B + \sin(\theta_i)\,z_B \le \tan(\alpha_c)\,y_B$ with $\theta_i = 2\pi(i-1)/n_f$, and the floor is $y_B \ge y_{\min}$.  In LVLH coordinates:
\begin{equation}
A_c\,R_z(-\omega_t t)\,\mathbf{p}_L \le b_c.
\label{eq:los_lvlh}
\end{equation}
Since the MPC operates in $\mathcal{F}_B$, constraints~\eqref{eq:los_3d} are time-invariant within the QP.


%%=========================================================================
\section{Problem Formulation}
\label{sec:problem}
%%=========================================================================

At each control step, the MPC solves over a receding horizon of $N_p$ steps:
\begin{subequations}
\label{eq:mpc}
\begin{align}
\min_{\mathbf{u}_{0:N_p\!-\!1}} \; J &= \sum_{j=0}^{N_p-1}\!\Big[ \|\hat{\mathbf{x}}_j - \mathbf{x}^{\mathrm{ref}}\|_Q^2 + \|\mathbf{u}_j\|_{R_u}^2 + \|\Delta\mathbf{u}_j\|_{R_{\Delta u}}^2 \Big] \notag\\
& \quad + \|\hat{\mathbf{x}}_{N_p} - \mathbf{x}^{\mathrm{ref}}\|_{Q_N}^2 \label{eq:cost} \\
\text{s.t.} \quad & \hat{\mathbf{x}}_{j+1} = A_d\,\hat{\mathbf{x}}_j + B_d\,\mathbf{u}_j, \; j = 0,\ldots,N_p\!-\!1, \label{eq:dyn_con} \\
& \hat{\mathbf{x}}_0 = \mathbf{x}_k, \label{eq:ic_con} \\
& A_c\,[\hat{\mathbf{x}}_j]_{1:3} \le b_c, \; j = 0,\ldots,N_p, \label{eq:los_con} \\
& -a_{\max}\mathbf{1} \le \mathbf{u}_j \le a_{\max}\mathbf{1}, \; j = 0,\ldots,N_p\!-\!1, \label{eq:input_con}
\end{align}
\end{subequations}
where $\Delta\mathbf{u}_j = \mathbf{u}_j - \mathbf{u}_{j-1}$ (with $\mathbf{u}_{-1}$ the previously applied input), and $Q = \mathrm{diag}(15, 30, 15, 1, 1, 1)$, $Q_N = 30Q$, $R_u = 10^{-2}I_3$.  The QP is solved by OSQP~\cite{Stellato2020} with warm-starting.

Two MPC configurations are used.  For single-scenario analysis: $N_p = 40$, $R_{\Delta u} = 10^4 I_3$, $T_{\mathrm{sim}} = 600$~s.  For Monte Carlo: $N_p = 20$, $R_{\Delta u} = \mathrm{diag}(10^5, 10^4, 10^5)$, $T_{\mathrm{sim}} = 400$~s.  The MC configuration uses asymmetric input-rate penalties and shorter horizon to balance fidelity against computational cost across 3\,900 simulations.


%%=========================================================================
\section{Guidance Architecture}
\label{sec:guidance}
%%=========================================================================

The guidance system is a single receding-horizon MPC controller operating in the target-body frame at every control step~$k$:
\begin{enumerate}
\item \textbf{Linearise.} Finite-difference Jacobians $(A_{d,k}, B_{d,k})$ are computed about the current state and input via the nonlinear propagation pipeline.
\item \textbf{Build and solve QP.} The cost~\eqref{eq:cost} with dynamics, LOS, and input-bound constraints is assembled and solved by OSQP~\cite{Stellato2020}.
\item \textbf{Apply.} Only $\mathbf{u}_0^*$ is applied after clamping to $[-a_{\max}, a_{\max}]$ per axis (receding horizon).
\item \textbf{Propagate.} The nonlinear truth model advances both spacecraft in ECI; the relative state is re-projected into $\mathcal{F}_B$.
\end{enumerate}
The reference trajectory is an exponential approach along the docking axis: $y_{\mathrm{ref}}(t) = y_{\mathrm{end}} + (y_0 - y_{\mathrm{end}})e^{-t/\tau}$.  No separate PD controller or regime switching is used.


%%=========================================================================
\section{Safe-Start Region Analysis}
\label{sec:reachability}
%%=========================================================================

\subsection{Directional Per-Constraint Erosion}
Body-frame rotation creates apparent velocity $\mathbf{v}_{\mathrm{rot}} = [\omega_t y_B, -\omega_t x_B, 0]^\top$ for an inertially-stationary chaser.  The margin consumed before braking is:
\begin{equation}
\delta_i = \frac{(\dot{s}_i^-)^2}{2\,a_{\max}}, \quad \dot{s}_i = -\mathbf{a}_i^\top \mathbf{v}_{\mathrm{rot}},
\label{eq:erosion}
\end{equation}
where $\dot{s}_i^- = \min(0, \dot{s}_i)$ is the negative part of the constraint-slack rate.

\subsection{Synchronisation Range Bound}
At range $r$, the apparent rotational speed is $v_{\mathrm{rot}} = \omega_t r$.  Requiring $d_{\mathrm{brake}} = \omega_t^2 r^2 / (2a_{\max}) < r$ gives:
\begin{equation}
r < r_{\mathrm{sync}} = \frac{2\,a_{\max}}{\omega_t^2}.
\label{eq:rsync}
\end{equation}

\subsection{Hierarchy of Certified Feasibility Sets}
\label{sec:hierarchy}

We distinguish three analytical feasibility regions, ordered by increasing conservatism, plus an independent empirical validation set:

\begin{enumerate}
\item \textbf{Nominal deterministic certified region} $\mathcal{X}_{\mathrm{nom}}$: the analytically computed inner approximation assuming perfect model knowledge and no process disturbance. Uses the directional per-constraint erosion~\eqref{eq:erosion} and synchronization range bound~\eqref{eq:rsync} to certify open-loop feasibility.

\item \textbf{Stochastic chance-constrained certified region} $\mathcal{X}_{\mathrm{stoch}}$: tightens each constraint by the quantile $\Phi^{-1}(1-\alpha/n_c) \cdot \sigma_i$ of the accumulated Gaussian process noise, guaranteeing constraint satisfaction with probability $\geq 1-\alpha$. With $\alpha=0.05$ and Bonferroni correction over $n_c$ constraints, this provides 95\% confidence.

\item \textbf{Robust bounded-disturbance certified region} $\mathcal{X}_{\mathrm{rob}}$: tightens each constraint by the worst-case accumulated disturbance support function $\sum_{j=0}^{N-1} \max_{\mathbf{w}\in\mathcal{W}} \mathbf{a}_i^\top A^j \mathbf{w}$, guaranteeing feasibility for \emph{all} disturbance realizations $\mathbf{w}_k \in \mathcal{W}$ over the analysis horizon.

\item \textbf{Empirical Monte Carlo set} $\mathcal{X}_{\mathrm{MC}}$: the set of initial positions from which full closed-loop simulation (with nonlinear ECI truth dynamics and the MPC controller) successfully completes the approach without LOS violation.  $\mathcal{X}_{\mathrm{MC}}$ is \emph{not} analytically nested with the three sets above; it serves as an independent empirical benchmark.
\end{enumerate}

By construction, the analytical sets satisfy the inclusion relation:
\begin{equation}
\mathcal{X}_{\mathrm{rob}} \subseteq \mathcal{X}_{\mathrm{stoch}} \subseteq \mathcal{X}_{\mathrm{nom}}.
\label{eq:hierarchy}
\end{equation}

This nesting reflects increasing conservatism:
\begin{itemize}
\item \emph{Nominal} assumes perfect model fidelity and zero disturbance.
\item \emph{Stochastic} guarantees constraint satisfaction with probability $\geq 1-\alpha$ under Gaussian process noise.
\item \emph{Robust} guarantees constraint satisfaction for \emph{all} bounded disturbances in a compact set $\mathcal{W}$.
\item \emph{Monte Carlo} estimates empirical closed-loop feasibility; it may be smaller or larger than the nominal set because the MPC controller may fail where open-loop analysis succeeds.
\end{itemize}

The analytical regions ($\mathcal{X}_{\mathrm{nom}}$, $\mathcal{X}_{\mathrm{stoch}}$, $\mathcal{X}_{\mathrm{rob}}$) are computed via the erosion model~\eqref{eq:erosion} with additional constraint tightening:
\begin{equation}
s_i(\mathbf{x}) - \delta_i^{\mathrm{nom}} - \Delta_i^{\mathrm{noise}} > 0, \quad \forall\, i = 1,\ldots,n_c,
\label{eq:tightened_erosion}
\end{equation}
where $\delta_i^{\mathrm{nom}}$ is the nominal directional erosion from~\eqref{eq:erosion} and $\Delta_i^{\mathrm{noise}}$ is the method-specific tightening (zero for nominal, $z_{1-\alpha/n_c} \sigma_i$ for stochastic, $\sum_j h_{\mathcal{W}}(\mathbf{a}_i^\top A^j)$ for robust).

Fig.~\ref{fig:hierarchy} illustrates the nested structure for representative cases. The progressive shrinkage from $\mathcal{X}_{\mathrm{nom}}$ to $\mathcal{X}_{\mathrm{rob}}$ quantifies the ``price of certification'' --- the reduction in operational region required for formal safety guarantees under increasingly stringent assumptions.


\subsection{Illustrative Reachability Example: Double Integrator}
\label{sec:example_di}

To illustrate the reachability concepts before applying them to the CWH dynamics, we consider a 2D double integrator:
\begin{equation}
\mathbf{x}_{k+1} = \underbrace{\begin{bmatrix} 1 & 1 \\ 0 & 1 \end{bmatrix}}_{A}\,\mathbf{x}_k + \underbrace{\begin{bmatrix} 0.5 \\ 1 \end{bmatrix}}_{B}\,u_k + \mathbf{w}_k,
\end{equation}
with state constraints $|x_1| \le 5$, $|x_2| \le 3$ and input bound $|u| \le 1$.

Fig.~\ref{fig:example_overlay} shows the backward reachable sets (safe-start regions) for $N=5$ steps to a target set near the origin, computed under three assumptions:
\begin{itemize}
\item \textbf{Nominal} (green): no disturbance, deterministic guarantee.
\item \textbf{Stochastic} (blue): Gaussian noise $\mathbf{w}_k \sim \mathcal{N}(0, W)$, chance constraints with $\alpha=0.05$.
\item \textbf{Robust} (purple): bounded disturbance $\|\mathbf{w}_k\|_\infty \le w_{\max}$, worst-case guarantee.
\end{itemize}

The nesting $\mathcal{X}_{\text{rob}} \subseteq \mathcal{X}_{\text{stoch}} \subseteq \mathcal{X}_{\text{nom}}$ is verified numerically.  This same hierarchy, applied to the CWH dynamics with rotating LOS constraints, yields the feasibility certification results in Section~\ref{sec:hierarchy}.

\begin{figure}[H]
\centering
\includegraphics[width=0.85\columnwidth]{figures/overlay_comparison.pdf}
\caption{Nested backward reachable sets for a 2D double integrator ($N=5$).  The robust set (purple) is the smallest but provides the strongest guarantee; the nominal set (green) is the largest but assumes zero disturbance.}
\label{fig:example_overlay}
\end{figure}



%%=========================================================================
\section{Monte Carlo Validation}
\label{sec:mc}
%%=========================================================================

For each of the $5 \times 4 = 20$ $(\omega_t, a_{\max})$ combinations:

\begin{enumerate}
\item \textbf{IC sampling.}  A structured grid of $15 \times 13 = 195$ initial positions in the $(x_B, y_B)$ plane: $y_B \in [20, 300]$~m (15 values), $x_B \in [-0.95\tan(30^\circ)y_B,\; 0.95\tan(30^\circ)y_B]$ (13 values per $y_B$).  Initial velocity is zero in $\mathcal{F}_B$.

\item \textbf{Closed-loop simulation.}  Each IC is simulated for 400~s using nonlinear ECI truth dynamics~\eqref{eq:eci_eom}, quaternion propagation~\eqref{eq:quat_prop}, online linearisation, QP~\eqref{eq:mpc} with MC configuration, and body-frame extraction~\eqref{eq:pos_transform}--\eqref{eq:vel_transform}.

\item \textbf{Classification.}  \emph{Feasible}: zero LOS violations (tolerance $\epsilon = 10^{-3}$~m) over $T_{\mathrm{sim}}$.  \emph{Infeasible}: any violation exceeds $\epsilon$, OSQP reports infeasibility, or early termination.

\item \textbf{Interpolation.}  Binary outcomes are mapped to the $300^2$ analytical grid via natural-neighbour interpolation with nearest-neighbour extrapolation.
\end{enumerate}

Each of the 3\,900 simulations is independent, enabling \texttt{parfor} parallelisation.  Total campaign: 2--4~hours on an 8-core workstation.


%%=========================================================================
\section{Results}
\label{sec:results}
%%=========================================================================

\subsection{Nominal Reachability Maps}
Table~\ref{tab:reachability_sweep} reports the safe fraction of the LOS cone for each combination, using the forward erosion criterion~\eqref{eq:erosion} with the synchronisation range bound~\eqref{eq:rsync}.

\begin{table}[H]
\centering
\caption{Safe fraction of the body-frame LOS cone (\%) for various tumble-rate and thrust-authority combinations. Criteria: directional per-constraint erosion~\eqref{eq:erosion} and synchronisation range bound~\eqref{eq:rsync}. Grid: $300 \times 300$ points.}
\label{tab:reachability_sweep}
\begin{tabular}{c|cccc}
\hline
$\omega_t$ (deg/s) & $a_{\max}=0.20$ & $0.10$ & $0.05$ & $0.02$ \\
\hline
1 & 80.1 & 66.6 & 45.0 & 7.2 \\
2 & 45.0 & 11.3 & 2.8 & 0.4 \\
3 & 8.9 & 2.2 & 0.6 & 0.1 \\
4 & 2.8 & 0.7 & 0.2 & 0.0 \\
5 & 1.1 & 0.3 & 0.1 & 0.0 \\
\hline
\end{tabular}
\end{table}


The safe fraction decreases rapidly with tumble rate due to the $\omega_t^{-2}$ dependence of $r_{\mathrm{sync}}$ and the quadratic growth of the erosion term.  At $\omega_t = 1$~deg/s with $a_{\max} = 0.20$~m/s$^2$, 80.1\% of the cone is certified safe; at $\omega_t = 5$~deg/s with $a_{\max} = 0.02$~m/s$^2$, the safe region shrinks to near zero.

\subsection{Forward vs. Backward Analysis}

The backward viability kernel (discrete LP, $N_{\mathrm{back}} = 20$ steps) is substantially larger than the erosion-based estimate.  At $\omega_t = 1$~deg/s, $a_{\max} = 0.20$~m/s$^2$: 80.1\% forward vs.\ 98.1\% backward---an 18-pp gap quantifying the conservatism of single-step braking.  At $\omega_t = 3$~deg/s, $a_{\max} = 0.20$~m/s$^2$: 8.9\% vs.\ 73.9\% (65-pp gap).

\subsection{Feasibility Hierarchy}

\begin{figure}[H]
\centering
\includegraphics[width=\columnwidth]{figures/comparison_grid.png}
\caption{Nested feasibility hierarchy for all 20 parameter combinations. Nominal (green) $\supseteq$ stochastic (blue) $\supseteq$ robust (purple).}
\label{fig:hierarchy}
\end{figure}

The inclusion $\mathcal{X}_{\mathrm{rob}} \subseteq \mathcal{X}_{\mathrm{stoch}} \subseteq \mathcal{X}_{\mathrm{nom}}$ is verified numerically at every grid point across all 20 combinations with zero violations.  The stochastic set uses Bonferroni correction with $\alpha = 0.05$ over $n_c = 9$ constraints; the robust set uses support-function tightening for bounded disturbances.

\subsection{Monte Carlo Feasibility Maps}

\begin{figure}[H]
\centering
\includegraphics[width=\columnwidth]{figures/fig_mc_sweep_grid.png}
\caption{MC feasibility sweep: 195 ICs per scenario.  Green: feasible; red: infeasible.}
\label{fig:mc_grid}
\end{figure}

\subsection{Single-Scenario Overlay}

\begin{figure}[H]
\centering
\includegraphics[width=0.95\columnwidth]{figures/single_w2_a0.10.png}
\caption{Overlay for $\omega_t = 2$~deg/s, $a_{\max} = 0.10$~m/s$^2$: MC $\supseteq$ nominal $\supseteq$ stochastic $\supseteq$ robust.}
\label{fig:single}
\end{figure}

\subsection{Closed-Loop Sweep}
\begin{table}[H]
\centering
\caption{Parameter sweep: approach outcome vs.\ tumble rate and thrust authority.}
\label{tab:sweep}
\begin{tabular}{c c c c c c c c}
\hline
$\omega_t$ (°/s) & $a_{\max}$ (m/s$^2$) & $r_{\mathrm{sync}}$ (m) & Hold & $r_{\min}$ (m) & $r_{\mathrm{final}}$ (m) & $\Delta v$ (m/s) & LOS viol. \\
\hline
1 & 0.10 & 656.6 & No & 0.0 & 2.4 & 29.5 & 441 \\
1 & 0.05 & 328.3 & No & 143.7 & 274.4 & 15.0 & 803 \\
1 & 0.02 & 131.3 & No & 145.3 & 275.9 & 6.0 & 2313 \\
2 & 0.10 & 164.1 & No & 143.0 & 191.7 & 30.0 & 2615 \\
2 & 0.05 & 82.1 & No & 141.9 & 153.4 & 15.0 & 2021 \\
2 & 0.02 & 32.8 & No & 111.8 & 123.9 & 6.0 & 1912 \\
3 & 0.10 & 73.0 & No & 138.2 & 392.8 & 30.0 & 2070 \\
3 & 0.05 & 36.5 & No & 111.4 & 142.8 & 15.0 & 2230 \\
3 & 0.02 & 14.6 & No & 67.1 & 202.6 & 6.0 & 1957 \\
4 & 0.10 & 41.0 & No & 113.2 & 127.6 & 30.0 & 2003 \\
4 & 0.05 & 20.5 & No & 67.3 & 79.8 & 15.0 & 2092 \\
4 & 0.02 & 8.2 & No & 62.1 & 161.3 & 6.0 & 2225 \\
5 & 0.10 & 26.3 & No & 75.0 & 150.6 & 30.0 & 1996 \\
5 & 0.05 & 13.1 & No & 63.9 & 97.2 & 15.0 & 2096 \\
5 & 0.02 & 5.3 & No & 51.1 & 149.0 & 6.0 & 2060 \\
\hline
\end{tabular}
\end{table}

Starts from outside $r_{\mathrm{sync}}$ produce LOS violations; starts within $r_{\mathrm{sync}}$ achieve \textbf{zero violations}, validating the reachability predictions.

\subsection{Computational Efficiency}

\begin{table}[H]
\centering
\caption{Computation times.}
\label{tab:timing}
\begin{tabular}{lc}
\hline
Method & Time \\
\hline
Forward nominal (20 combos) & 2.4 s \\
Backward LP (20 combos) & 46.0 s \\
Stochastic (20 combos) & 2.4 s \\
Robust (20 combos) & 2.2 s \\
\textbf{Total analytical} & \textbf{53 s} \\
\hline
Monte Carlo ($20 \times 195$ sims) & 2--4 h \\
\textbf{Speedup} & $\sim$\textbf{200$\times$} \\
\hline
\end{tabular}
\end{table}


%%=========================================================================
\section{Discussion}
\label{sec:discussion}
%%=========================================================================

\subsection{Price of Certification}
For $\omega_t = 1$~deg/s, $a_{\max} = 0.20$~m/s$^2$: robust covers 75.2\% (vs.\ 80.1\% nominal)---only 4.9~pp reduction.  For $\omega_t \ge 4$~deg/s, all regions $<$3\%, suggesting de-tumbling the target before approach.

\subsection{Universal Scaling Parameter}
The dimensionless ratio $a_{\max}/\omega_t^2$ (dimension: length, equals $r_{\mathrm{sync}}/2$) governs both erosion magnitude and safe-region extent.  All results collapse when plotted against this single parameter.

\subsection{CWH Along-Track Coupling}
The secular term $6n(s - n\tau)$ in the CWH state-transition matrix produces along-track drift proportional to radial offset. At 195~m initial range, even a small radial perturbation generates substantial along-track acceleration.  Implementations using double-integrator truth with reference blending show up to 80\% phantom $\Delta v$ from state teleportation.

\subsection{Erosion Conservatism}
The forward--backward gap quantifies single-step braking pessimism.  Optimal multi-constraint trajectories can exploit constraint margins that instantaneous braking analysis misses.

\subsection{Positioning Relative to Tube-Based MPC and RG-OCP}
Benchmarking against tube-based robust MPC for tumbling targets~\cite{Dong2020,Quartullo2024} and the reachability-guaranteed OCP of~\cite{Faraci2025} reveals complementary guarantees.  Tube MPC embeds robustness within the controller via mRPI-based constraint tightening; the RG-OCP guarantees terminal interception under orientation uncertainty; our framework certifies the approach corridor---which initial conditions permit constraint-satisfying trajectories.  At the Faraci benchmark parameters ($\omega_t = 4$~deg/s, $T_f = 30$~s), the erosion-based analysis achieves a $\sim$34$\times$ speedup over the NLP-based RG-OCP (77~ms vs.\ 2658~ms per combination), enabling rapid parametric sweeps for pre-mission planning.


%%=========================================================================
\section{Conclusions}
\label{sec:conclusions}
%%=========================================================================

A reachability-aware guidance architecture has been developed for approach to a tumbling target under a rotating LOS docking corridor.  The main findings are:
\begin{enumerate}
\item The analytical hierarchy $\mathcal{X}_{\mathrm{rob}} \subseteq \mathcal{X}_{\mathrm{stoch}} \subseteq \mathcal{X}_{\mathrm{nom}}$ verified across all 20 combinations with zero point-wise violations.  The independent Monte~Carlo set $\mathcal{X}_{\mathrm{MC}}$ validates the analytical predictions.
\item $a_{\max}/\omega_t^2$ is the universal scaling parameter for rotating-corridor feasibility.
\item Analytical certification: 53~s vs.\ 2--4~h MC ($\sim$200$\times$ speedup), enabling on-board replanning.
\item Nonlinear truth propagation reveals double-integrator artefacts (up to 80\% phantom $\Delta v$).
\item Approach within $r_{\mathrm{sync}} = 2a_{\max}/\omega_t^2$ achieves zero LOS violations.
\end{enumerate}

Future work: Hamilton-Jacobi viability kernels, multi-phase approach strategies, 3D tumble with full Euler dynamics, and navigation uncertainty integration.

\section*{Acknowledgements}
The author thanks the open-source community for the tools used in this work.

\bibliography{references}

\end{document}
