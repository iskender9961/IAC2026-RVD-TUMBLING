\subsection{Hierarchy of Certified Feasibility Sets}
\label{sec:hierarchy}

We distinguish four feasibility regions, ordered by increasing conservatism:

\begin{enumerate}
\item \textbf{Empirical Monte Carlo safe region} $\mathcal{X}_{\mathrm{MC}}$: the set of initial positions from which full closed-loop simulation (with nonlinear ECI truth dynamics and the MPC controller) successfully completes the approach without LOS violation. This is the least conservative but provides no formal guarantees.

\item \textbf{Nominal deterministic certified region} $\mathcal{X}_{\mathrm{nom}}$: the analytically computed inner approximation assuming perfect model knowledge and no process disturbance. Uses the directional per-constraint erosion~\eqref{eq:erosion} and synchronization range bound~\eqref{eq:rsync} to certify that the controller can maintain all LOS constraints.

\item \textbf{Stochastic chance-constrained certified region} $\mathcal{X}_{\mathrm{stoch}}$: tightens each constraint by the quantile $\Phi^{-1}(1-\alpha/n_c) \cdot \sigma_i$ of the accumulated Gaussian process noise, guaranteeing constraint satisfaction with probability $\geq 1-\alpha$. With $\alpha=0.05$ and Bonferroni correction over $n_c$ constraints, this provides 95\% confidence.

\item \textbf{Robust bounded-disturbance certified region} $\mathcal{X}_{\mathrm{rob}}$: tightens each constraint by the worst-case accumulated disturbance support function $\sum_{j=0}^{N-1} \max_{\mathbf{w}\in\mathcal{W}} \mathbf{a}_i^\top A^j \mathbf{w}$, guaranteeing feasibility for \emph{all} disturbance realizations $\mathbf{w}_k \in \mathcal{W}$ over the analysis horizon.
\end{enumerate}

By construction, these sets satisfy the inclusion relation:
\begin{equation}
\mathcal{X}_{\mathrm{rob}} \subseteq \mathcal{X}_{\mathrm{stoch}} \subseteq \mathcal{X}_{\mathrm{nom}} \subseteq \mathcal{X}_{\mathrm{MC}}.
\label{eq:hierarchy}
\end{equation}

This nesting reflects increasing conservatism and decreasing modeling assumptions:
\begin{itemize}
\item \emph{Nominal} assumes perfect model fidelity and zero disturbance.
\item \emph{Stochastic} guarantees constraint satisfaction with probability $\geq 1-\alpha$ under Gaussian process noise.
\item \emph{Robust} guarantees constraint satisfaction for \emph{all} bounded disturbances in a compact set $\mathcal{W}$.
\item \emph{Monte Carlo} estimates empirical feasibility through exhaustive simulation; it is the most permissive but carries no analytic certification.
\end{itemize}

The analytical regions ($\mathcal{X}_{\mathrm{nom}}$, $\mathcal{X}_{\mathrm{stoch}}$, $\mathcal{X}_{\mathrm{rob}}$) are computed via the erosion model~\eqref{eq:erosion} with additional constraint tightening:
\begin{equation}
s_i(\mathbf{x}) - \delta_i^{\mathrm{nom}} - \Delta_i^{\mathrm{noise}} > 0, \quad \forall\, i = 1,\ldots,n_c,
\label{eq:tightened_erosion}
\end{equation}
where $\delta_i^{\mathrm{nom}}$ is the nominal directional erosion from~\eqref{eq:erosion} and $\Delta_i^{\mathrm{noise}}$ is the method-specific tightening (zero for nominal, $z_{1-\alpha/n_c} \sigma_i$ for stochastic, $\sum_j h_{\mathcal{W}}(\mathbf{a}_i^\top A^j)$ for robust).

Fig.~\ref{fig:hierarchy} illustrates the nested structure for representative cases. The progressive shrinkage from $\mathcal{X}_{\mathrm{MC}}$ to $\mathcal{X}_{\mathrm{rob}}$ quantifies the ``price of certification'' --- the reduction in operational region required for formal safety guarantees under increasingly stringent assumptions.
